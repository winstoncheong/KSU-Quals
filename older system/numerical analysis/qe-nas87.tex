% latex file
\def\hcorrection#1{\advance\hoffset by #1 }
\def\vcorrection#1{\advance\voffset by #1 }

\documentclass{article}
\usepackage{my,amsxtra,amssymb,amsthm}

\vcorrection{-1.0in}
\hcorrection{-0.8in}
\textwidth 6.0in
\textheight 9.0in

\def\R{\mathbb R}
\def\C{\mathbb C}
\def\N{\mathbb N}
\def\Z{\mathbb Z}
\def\Q{\mathbb Q}

\begin{document}
%\begin{Large}






\begin{center}\begin{LARGE}
{\bf Numerical Analysis Qualifying Exam}\\ 
{\bf Spring 1987}\\ \end{LARGE}
\end{center}
\vspace{0.1in}
\noindent\hrulefill\\

Hand in at most ten problems. You must work at least one from each of the
five sections.

\begin{description}
\item[I.]
Differentialtion, Integration, and General Topics.

\item[\quad] 1.
A centered difference method for approximating a second derivative uses
$$\frac{f(x+h) - 2f(x) + f(x-h)}{h^2},$$
$h>0$, for the approximation. If $f^{(4)}$ is defined and continuous on an
interval $[a,b]$ containing $x-h$ and $x+h$, show that the error in the
approximation is given by
$$-\frac{h^2}{12} f^{(4)} (c)$$
for some $c\varepsilon (x-h, x+h)$.

\item[\quad] 2.
Let $E(f) = \sum^n_{k=0} \alpha_k f(x_k)$ be a simple quadrature formula
which uses $n+1$ distinct nodes. If $E$ has degree of procision at least $n$,
show that $E$ is an interpolatory quadrature.

\item[\quad] 3.
Suppose we are given a program which approximates $\int^a_b f(x) dx$ using
composite Simpson's rule. The user supplies the integrand $f(x)$, the
limits $a$, $b$, and the number of panels $N$. The program runs on a computer
which uses nine decimal digit floating point calculations. If we seek the
value of $\int^1_0 \sqrt {x} \exp xdx$ we get the following results:
$$\begin{array}{cc}
        N \hbox{\ (Panels)} & \hbox{Approximate\ } \hbox{\ Integral} \\
        5 & 1.253107 \\
        10 & 1.254730 \\
        20 & 1.255311 \\
        40 & 1.255517 \\
        100 & 1.255601 \\
        400 & 1.255627 \\
        800 & 1.255629
        \end{array} $$
The actual value of the integral, correct to 7 figures is 1.255630.

\item[\qquad] (a)
Why should we have expected possible trouble in directly using Simpson's
rule on this integral?

\item[\qquad] (b)
What should be done to use the program more efficiently to calculate
the value of this integral?

\item[\quad] 4.
Explain how to calculate accurately the value of
$$f(x) = \frac{6 \sin x-6x+x^3}{x^5} $$
for very small positive $x$. Why is there any difficulty at all?

\item[II.]
Root Finding.

\item[\quad] 5. (a)
Describe Newton's method for functions of several variables. State without
proof the quadratic convergence theorem for this method. (Use the
asymptotic convergence rate difinition).

\item[\qquad] (b)
Let
$$f(x) = \begin{cases}
                \sqrt{x} & \hbox{if\ }  x\geq 0 \\
                -\sqrt{x} & \hbox{if\ }  x < 0
                \end{cases}$$

Apply Newton's method for any starting value and explain the results in
light of the convergence theorem you stated (a).

\item[\quad] 6. (a)
Describe Aitken's $\delta^2$-method of accelerating the convergence of a
sequence.

\item[\qquad] (b)
Let $\{x_n\}$ converge to $x$ with $x_n \neq x$ for all $n= 1,2, \dots$.
Show that  if the convergence is linear, then
$$\lim_{n \to +\infty} \frac{x^\prime_n - x}{x_n - x} = 0$$
where $\{x^\prime_n\}$ is the associated accelerated sequence.

\item[\quad] 7.
Give an example to show that the process of determining a root of a polynomial
is, in general, ill-conditioned.

\item[III.]
Approximation Theory.

\item[\quad] 8. (a)
Describe the interpolating polynomial $P_n$ for a real valued function $f$
and distinct points $\{x_0, \dots, x_n\}$ in the domain of $f$.

\item[\qquad] (b)
Let $f$ be infinitely differentiable on $[a,b]$ and suppose there exists a
real number $M$ such that $|f^{(n)} (x) \leq M$ for all
$x\varepsilon [a,b]$ and all $n \geq 0$. For $n \geq 0$, let $P_n$
interpolate $f$ at some set of $n+1$ distinct points in
$[a,b]$. Show that $P_n \to f$ uniformly on $[a,b]$.

\item[\qquad] (c)
For an arbitray function $f$ defined on $[a,b]$ and for evenly spaced nodes
$x_k = a + \frac{k}{n} (b-a)$, $0 \leq k \leq n$ in $[a,b]$, let $P_n$
interpolate $f$. Does $p_n \to f$ uniformly on $[a,b]$? State the
appropriate theorem or give a counterexample.
You need not give a proof for this part.

\item[\quad] 9.
Of all polynomials of degree $\leq 3$, find the one which best approximates
$p(x) = 2x^4 - 3x^2 + x+ 1$ on the interval $[-1, 1]$ with respect to the
uniform norm $\parallel \quad \parallel_\infty$. Use the fact that if
$q_n(x)$ is the polynomial in $P_n$ which best approximates
$x^{n+1}$ on $[-1,1]$, then $x^{n+1} - q_n (x) = 2^nT_{n+1} (x)$
where $T_{n+1} (x)$ is the Chebyshev polynomial.

\item[\quad] 10.
Let $V$ be a real vector space, $W$ a finite dimensional subspace of $V$
having dimension $n$, and let $<,>$ be a symmetric, positive semidefinite,
bilinear form on $V$ such that $<,>$ is positive definite on $W$.
Let $p_1, \dots, p_n$ be an orthonormal basis for $W$ and let
$\parallel v \parallel = \sqrt{<v,v>}$ for $v \varepsilon V$.

\item[\quad] (a)
Let $v \varepsilon V$ and let $w = \sum^n_{i=1} \beta_i p_i \varepsilon W$.
Show that
$$\parallel v-w^2 \parallel^2 = \parallel v \parallel^2 + \sum^n_{j=0}
  \left( \beta_j -<v,p_j> \right)^2 - \sum^n_{j=0}
  <v, p_j>^2.$$

\item[\quad] (b)
Given $v\varepsilon V$, show that there is a unique $w \varepsilon W$ that
minimizes $\parallel v-w \parallel$ and that it is given by
$$w = \sum^n_{j=1} <v, p_j>p_j.$$

\item[IV.]
Linear Algebra.

\item[\quad] 11.
Assume that the equation $x=Ax_b$ has a unique solution and consider the
iterative scheme
$$x_{i+1} = Ax_i + b$$
where $A$ is an $n$ by $n$ complex matrix and $x$ and $b$ are complex
$n$-vectors.

\item[\qquad] (a)
Prove that if the sequence $\{x_i\}$ converges for an arbitrary starting
vector $x_0$, then $\tau_\sigma (A) < 1$, where $\tau_\sigma (A)$ is
the spectral radius of $A$.

\item[\qquad] (b)
Prove that if $\tau_\sigma (A) < 1$, then $\{x_i\}$ converges for all starting
values $x_0$. (Hint: Use the spectral radius formula.)

\item[\quad] 12.
Outline in detail the Q-R method with Francis Q-R step. State, without
proof, any relevant theorems associated with this method.

\item[\quad] 13.
Does the Gauss-Seidel method converget for the linear equation
$$\left[\begin{array}{cccc}
        4&2&1&1 \\
        1&2&1&0 \\
        0&2&3&1 \\
        1&1&1&4
        \end{array}
        \right]
  \left[\begin{array} {c}
        x_1 \\
        x_2 \\
        x_3 \\
        x_4
        \end{array}
        \right]
   = \left[\begin{array}{c}
        2 \\
        1 \\
        3 \\
        5
        \end{array}
        \right]
         ?$$

\item[V.]
Differential Equations.

\item[\quad] 14.
Let $f: [a,b] \to R$ satisfy a Lipschitz condition,
$$|f(x,y) - f(x,y^\prime)| \leq L|y-y^\prime|$$
for all $x\varepsilon [a,b]$ and $y \varepsilon R$. Show that Euler's
method for the numerical solution of the initial value problem
$$y^\prime = f(x,y)$$
$$y(0) = y_0$$
is stable. Begin by stating the meaning of ``stable" in this context.

\item[\quad] 15.
Explain the meaning of the term ``parasitic solution" asd it relates to the
numerical solution of ordinary differential equations. Relate it to the
concept of stability and to the root condition.




\end{description}    
%\end{Large}
\end{document}














