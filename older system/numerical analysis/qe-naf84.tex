% latex file
\def\hcorrection#1{\advance\hoffset by #1 }
\def\vcorrection#1{\advance\voffset by #1 }

\documentclass{article}
\usepackage{my,amsxtra,amssymb,amsthm}

\vcorrection{-1.0in}
\hcorrection{-0.8in}
\textwidth 6.0in
\textheight 9.0in

\def\R{\mathbb R}
\def\C{\mathbb C}
\def\N{\mathbb N}
\def\Z{\mathbb Z}
\def\Q{\mathbb Q}

\begin{document}
%\begin{Large}


\begin{center}\begin{LARGE}
{\bf Numerical Analysis Qualifying Exam}\\ 
{\bf Fall 1984}\\ \end{LARGE}
\end{center}
\vspace{0.1in}
\noindent\hrulefill\\

\begin{description}
\item[1.]
You are responsible for writing an algorithm which will take, as input, any
number $x \varepsilon [-20, 20]$ and return an accurately computed value of
$$f(x) = \frac{1+x-e^x}{x^2}.$$
Explain, in detail, how you would do this.

\item[2.]
We want to solve the matrix equation
$$Ax=b \eqno{(*)}$$
by the Jacobi iteration scheme. Recall that the equation is first written
in the form
$$x=Bx+c.$$

\item[\quad] (a)
Define the matrix $B$ and the vector $c$ for the Jacobi iteration.

\item[\quad] (b)
Let $x^{(0)}$ be an initial guess. The Jacobi iteration is given by
$$x^{(k+1)} = Bx^{(k)} + c.$$

If $x$ is the exact solution of the equation $(*)$ show that
$x^{(k)} - x = B^k \left(x^{(0)} - x \right)$. Then state condition(s) on
$B$ which imply that $x^{(k)} \to x$ for every choice of
$x^{(0)}$.

\item[\quad] (c)
State condition(s) on $A$ which imply that $B$ satisfies the condition(s)
you gave in $(b)$.

\item[3.]
Suppose the only rule we have for evaluating integrals is the composite
Simpson's Rule. Tell how to treat each of the following integrals so that it
can be evaluated by Simpson's Rule without the risk of introducing
large errors.

\item[\quad] (a)
$\int^2_{-2} \frac{e^{x^2}}{\sqrt{4-x^2}} dx$

\item[\quad] (b)
$\int^\infty_0 e^{-x^2} \frac{dx}{1+x^2}$

\item[\quad] (c)
$\int^4_0 \frac{e^{-x}}{\sqrt{x}} dx$

\item[4.]
Use Householder's method to reduce the matrix
$$\left[\begin{array}{ccc}
        2&3&4 \\
        3&1&-1 \\
        4&-1&1
        \end{array}
        \right]$$
to tridiagonal form.

\item[5.]
Fit a cubic spline through the three points (1,0), (2,1) and (3,0), using
the natural boundary conditions
$$S^{\prime \prime} = 0 = S^{\prime \prime} (3).$$

\item[6.]
Let $I_n (f) = \sum^n_{i=1} A_i f(x_i)$ be the $n$-point Gaussian quadrature
formula for approximating
$$\int^1_{-1} f(x) dx.$$

\item[\quad] (a)
How are the nodes $x_1, \dots, x_n$ chosen?

\item[\quad] (b)
Show that the interpolatory quadrature formula with the nodes
$x_1, \dots, x_n$ given in (a) has degree of precision $2n-1$. (Hint: Use
the formula
$$f(x) = p_{n-1} (x) + f[x_1, \dots, x_n, x] (x-x_1) \dots (x-x_n).)$$

\item[\quad] (c)
Of course, the interpolatory quadrature  with nodes $x_1, \dots, x_n$ is
the $n$-point Gaussian formula. Use this fact to give the explicit formula
for the 2-point Gaussian formula.

\item[7.]
Find the polynomial of degree $\leq 1$ which gives the best uniform
approximation to $f(x) = x^2$ on the interval $[0,1]$. Be sure to
{\it justify} your answer.

\vspace{.5in}

Some possibly useful information:

Householder: $x \neq 0$, $n$ dimensional vector
$$S= \left(x^2_1 + \dots + x^2_n \right)^{\frac{1}{2}}$$
$$k= \pm S$$
$$K^2 = \frac{1}{2} (k^2 - kx_1)$$
$$u=x-ke_1$$
$$P=I - \frac{uu^*}{2K^2}$$

Recursions for orthogonal polynomials

(Legendre)
$$P_{r+1} (x) = \frac{2r+1}{r+1} xP_r(x) - \frac{r}{r+1} P_{r-1} (x)$$
$$P_0 (x) = 1, P_1 (x) = x$$

(Chebychev)
$$T_{r+1} (x) = 2x T_r (x) - T_{r-1} (x)$$
$$T_0 (x) = 1, T_1 (x) = x$$







\end{description}    
%\end{Large}
\end{document}














