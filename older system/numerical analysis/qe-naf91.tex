% latex file
\def\hcorrection#1{\advance\hoffset by #1 }
\def\vcorrection#1{\advance\voffset by #1 }

\documentclass{article}
\usepackage{my,amsxtra,amssymb,amsthm}

\vcorrection{-1.0in}
\hcorrection{-0.8in}
\textwidth 6.0in
\textheight 9.0in
\begin{document}
%\begin{Large}






\begin{center}\begin{LARGE}
{\bf Numerical Analysis Qualifying Exam}\\ 
{\bf Fall 1991}\\ \end{LARGE}
\end{center}
\vspace{0.1in}
\noindent\hrulefill\\

\begin{description}
\item[1.]
A rectagular parallelpiped has sides 31 mm (millimeter), 42 mm and 53 mm,
measured only to the nearest millimeter. Give a practical upper and lower
bounds for the total surface area of the parallelpiped.

\item[2.]
The computation of the sequence $p_n = (1/3)^n$ is a stabel mathematical
problem. The sequence can also be generated by the following recurrence
relation:
$$p_0 = 1, p_1 = 1/3, p_n = \frac{5}{3} p_{n-1} - \frac{4}{9}
  p_{n-2}, n = 2,3, \dots$$
Indicate with proof if this recurrence relation is stable.

\item[3.]
Suppose that the equation $f(x) = 0$ can be rearranged as $x = h(x)$ for
some function $h \in C^2$.

\item[\quad] (1)
Show that if, for some non-zero value of a parameter $\omega$, the sequence
generated by
$$x_{n+1} = x_n + \omega [ h(x_n) - x_n]$$
converges to a number $\alpha$, then $\alpha$ must be a fixed point of $h$.

\item[\quad] (2)
Taking $g(x) = x + \omega [h(x) - x]$, deduce that the above iteration is
second order if
$$\omega = \frac{1}{1-h^\prime (\alpha)} \, , \quad h^\prime (\alpha) \neq 1$$

\item[\quad] (3)
Based on the discussion, can you suggest a practical way of choosing a
suitable $\omega$ during the iteration? (you do not need to give a proof
for it)

\item[4.]
Suppose $f(x)$ have an $(n+1)$st derivative in $[a,b]$ and $P_n(x)$ is the
interpolation polynomial with respect to $n+1$ distinct points
$x_i, i =0, 1, \dots, n,  x_i \in [a,b]$
(i.e. $P_n(x_i) = f(x_i)$). Show that for any $x \in [a,b]$, there exists a
$\xi = \xi (x)$, with
$$\min (x_0, x_1, \dots, x_n, x) < \xi < \max (x_0, x_1, \dots, x_n, x),$$
such that
$$f(x) - P_n(x) \equiv R_n(x) = \frac{\omega_n (x)}{(n+1)!}
  f^{(n+1)} (\xi).$$
where
$$\omega_n (x) \equiv (x - x_0) (x-x_1) \dots (x-x_n).$$

\item[5.]
Given the following theorem: a quadrature formula
$$I_n\{f\} = \sum^n_{j=1} \alpha_j f(x_j)
  \eqno{(1)} $$



\end{description}    
%\end{Large}
\end{document}














