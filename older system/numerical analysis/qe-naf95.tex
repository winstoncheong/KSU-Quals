% latex file
\def\hcorrection#1{\advance\hoffset by #1 }
\def\vcorrection#1{\advance\voffset by #1 }

\documentclass{article}
\usepackage{my,amsxtra,amssymb,amsthm}

\vcorrection{-1.0in}
\hcorrection{-0.8in}
\textwidth 6.0in
\textheight 9.0in
\begin{document}
%\begin{Large}






\begin{center}\begin{LARGE}
{\bf Numerical Analusis Qualifying Exam}\\ 
{\bf Fall 1995}\\ \end{LARGE}
\end{center}
\vspace{0.1in}
\noindent\hrulefill\\

\begin{description}

\item[1.]
The flow velocity in a cross section of a square duct is given by:
$$u(y,z) = \sum^\infty_{k=1,3,\dots} (-1)^{(k-1)/2} \left[ 1-
  \frac{\cosh (k \pi z / 2)}{\cosh (k \pi /2)} \right]
  \frac{\cos (k \pi y/2)}{k^3}, \cosh(x) = \frac{e^x + e^{-1}}{2}.
  \eqno{(1)}$$
The flow is in $x$-direction and $u$ is a function of $y,z$ in a cross
section, where $y,z$ are between $-1$ and $1$.

\item[\quad] (a)
Assume that the series is an alternating series for large $k$'s. Estimate
the number of terms $n$ used to approximate the series such that the error
is less than or equal to $10^{-10}$ (a reasonable estimate for the worst
case is enough).

\item[\quad] (b)
A program will cause overflows if magnitude of an intermediate result in the
computation is greater than a certian number. Overflow does not give a
numerical result. On the other hand, if the magnitude of an intermediate
result is less than a certain number, underflow occurs and the computer
will assign zero to the result. Reformulate the general term in the series to
avoid overflow (underflow is allowed).

\item[2.]
Let $f(x) = (x-a)^2 h(x)$, where $a$ is real and $h(x)$ is a smooth function
such that $h(a) \neq 0$. Show that the rate of convergence of Newton's method
to the root $a$ is only linear but the rate of convergence of the method
$$x_{k+1} = x_k - 2\frac{f(x_k)}{f^\prime (x_k)}$$
is quadratic. Extend this to the case that $f(x) = (x-a)^m h(x)$
($m$ is a positive integer) without proof.

\item[3.]
Use the Chebysev Equioscillation Theorem to prove the following result.

For a fixed integer $n>0$, consider the minimization problems for
$-1 \leq x \leq 1$:
$$\tau_n = \inf_{\deg (Q) \leq n-1} \parallel x^n + Q \parallel_\infty$$
with $Q(x)$ a polynomial. The minimum $\tau_n$ is attained uniquely
by letting
$$Q(x) = \frac{1}{2^{n-1}} T_n (x) - x^n$$
where $T_n(x)$ is the Chebysev polynomial
$$T_n(x) = \cos(n \cos^{-1} x).$$
The minimum is $\tau_n = \frac{1}{2^{n-1}}$.

\item[4.]
Describe the composite trapezoidal rule and composite Simpson's rule for
evaluating integrals $\int^b_a f(x) dx$ using the trapezoidal rule and
Simpson's rule:
$$\int^{x_0 +h}_{x_0} f(x) dx = \frac{h}{2} \left[ f(x_0) + f(x_0 + h) \right]
  - \frac{h^3}{12} f^{(2)} (\xi),$$
$$\int^{x_0 + 2h}_{x_0} f(x) dx = \frac{h}{3} \left[ f(x_0) + 4f (x_0 + h) +
  f(x_0 + 2h) \right] - \frac{h^5}{90} f^{(4)} (\xi),$$

\item[\quad] (1)
Derive an estimate for the error in the composite trapezoidal rule in terms
of the length of the subintervals into which $[a,b]$ is divided.

\item[\quad] (2)
Derive the asymptotic error formula for the composite Simpson's rule
$$E_n (f) \doteq - \frac{h^4}{180} \left[ f^{(3)} (b) - f^{(3)} (a) \right]$$

\item[5.]
Consider the numerical differentiation formula
$$f^\prime (x) \doteq \frac{f(z+h) - f(x-h)}{2h}$$

\item[\quad] (1)
Derive the formula with a remainder:
$$f^\prime (x) - \frac{f(x+h) - f(x-h)}{2h} = -\frac{h^2}{6} f^{(3)} (\xi)$$

\item[\quad] (2)
Assume $M$ is an upper bound of $f^{(3)} (x)$, $\varepsilon$ is an upper
bound of the absolute round off errors that occur when $f$ is evaluated.
Derive an upper bound for the total error. Discuss the effects of the
truncation error and the round off error.

\item[6.]
Prove that
$$\sigma_{\max} = \max_{x \in R^n, y \in R^m, x, y \neq 0}
  \frac{y^T Ax}{\parallel x \parallel_2 \parallel y \parallel_2}, \quad
  A \in R^{m \times n}$$
where $\sigma_{\max}$ is the largest singular value of $A$. (Hint: consider
the singular value decomposition of $A$)

\item[7.]
Let $A$ be a non-singular $n \times n$ matrix, consider the following
iterative scheme for solving the linear system $Ax=b$: Choose a non-singular
matrix $B$, let $x_0 = Bb$ and set
$$\Gamma_{n-1} = b - Ax_{n-1},$$
$$x_n = x_{n-1} + B \Gamma_{n-1}, n= 1,2,\dots$$

\item[\quad] (1)
Show that
$$\Gamma_n = (I - AB)^{n+1} b, n =0,1, \dots$$

\item[\quad] (2)
State and prove a criterion (in terms of $A, B$) for convergence of the
sequence $\{x_n\}$ to the solution of $Ax=b$.

\item[8.]
Assume that $w,u,v \in R^n$, and that $\parallel w \parallel_2 = 1$. What
are the eigenvalues, eigenvectors, and determinant of a Householder matrix
$I-2ww^T$? More generally, what can you say about the eigenvalues and
eigenvectors of $I+uv^T$ for given vectors $u$ and $v$? (The case
$v^T u =0$ is special.)




\end{description}    
%\end{Large}
\end{document}














