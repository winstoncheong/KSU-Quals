% latex file
\def\hcorrection#1{\advance\hoffset by #1 }
\def\vcorrection#1{\advance\voffset by #1 }

\documentclass{article}
\usepackage{my,amsxtra,amssymb,amsthm}

\vcorrection{-1.0in}
\hcorrection{-0.8in}
\textwidth 6.0in
\textheight 9.0in
\begin{document}
%\begin{Large}






\begin{center}\begin{LARGE}
{\bf Numerical Analysis Qualifying Exam}\\ 
{\bf Fall 1992}\\ \end{LARGE}
\end{center}
\vspace{0.1in}
\noindent\hrulefill\\

\begin{description}
\item[1.]
Consider evaluating $\exp(-x)$ for large $x$ by using the Taylor
approximation,
$$\exp (-x) \approx 1-x + \frac{x^2}{2!} - \dots + (-1)^n
  \frac{x^n}{n!}.$$
If one uses it to evaluate $\exp(-4)$, determine $n$ so that the Taylor
approximation error is less than .0005. Suppose one does the computation
using 4-digit rounding, what trouble will one encounter? How should
$\exp(-x)$ be evaluated for large positive values of $x$?

\item[2.]
Let $p$ be a root of the equation $f(x) = 0$, and assume that the Newton's
method for this equation converges to $p$. Assume also that $f(x)$ has
sufficiently high derivatives.

\item[\quad] (a)
If $p$ is a single root, show that the Newton's method converges
quadratically.

\item[\quad] (b)
If $p$ is a multiple root, show that it converges only lineraly.

\item[3.]
Suppose that the Lagrange interpolation formula for the function $f$ at
the $n+1$ distinct nodes $x_0, x_1, \dots x_n$ is given by
$$P_n(x) = \sum^n_{j=0} l_{j,n} (x) f(x_j),$$
where the Lagrange polynomial coefficients are given by
$$l_{j,n} (x) = \prod^n_{\substack {i =0 \\ i \neq j}}
  \frac{(x-x_i)}{(x_j - x_i)}.$$
Show that for any $n \geq 1$,
$$\sum^n_{j=0} l_{j,n} (x) = 1.$$

\item[4.]
The Trapezoidal Rule for an intergral $\int^b_a f(x)dx$ is based on linear
interpolation and hence has degree of precision (DOP) at least one. The
Simpson's Rule is based on quadratic interpolation with nodes $a, (a+b)/2, b$
and hence has DOP at least two. Based on the defination of DOP, explain,
without quoting the theorems on the accuracy of the quadrature rules, why
the Simpson's Rule has DOP three while the Trapezoidal Rule has DOP only one.
(Hint: first consider special $a,b$)

\item[5.]
Consider the numerical differentiation formula
$$f^\prime (a) = \frac{1}{2h} \left[ -3 f(a) + 4f(a+h) - f(a + 2h) \right]
  + \frac{h^2}{3} f^{\prime \prime \prime} (\xi),$$
where $a < \xi < a+2h$. If $f^{\prime \prime \prime}$ is bounded by $M$,
and the absolute error in evaluation of $f$ due to rounding is bounded by
$\varepsilon$, discuss the effect of the rounding error and find the best
choice for the value of $h$ (ignoring the rounding error in $h$ and all the
other rounding errors except the ones appear in the evaluations of $f$).

\item[6.]
Give an {\bf upper bound} for the relative error in the solution of the
system of linear equations
$$Ax=b$$
with symmetric matrix $A$ given by
$$A= \begin{bmatrix}
        6&1&2 \\
        1&6&2 \\
        2&2&8
       \end{bmatrix}$$
when the relative error in {\bf b} is less that $4 \cdot 10^{-4}$, i.e.
$$\frac{\parallel \delta b \parallel}{\parallel b \parallel}
< 4 \cdot 10^{-4}.$$
Use spectral norms, i.e., use
$$\parallel b \parallel = \left( \sum_j |b_j^2| \right)^{1/2}.$$

\item[7.]
Let $A = (a_{ij})$ be an $n \times n$ matrix. An iterative scheme for the
solution of the linear system $Ax = b$ is described by
$$\hbox{given\ } x_i^{(0)}, \quad i = 1, \dots, n;$$
$$a_{ii} y_i^{(k+1)} = b_i - \sum^{i-1}_{j=1} a_{ij}y_j^{(k+1)} -
  \sum^n_{j=i+1} a_{ij}x_j^{(k)}$$
$$x_i^{(k+1)} = \omega y_i^{(k+1)} + (1-\omega) x_i^{(k)}, \quad
    i=1, \dots, n; \quad k = 0,1, \dots $$

\item[\quad] (a)
    Write the iterative scheme in the form
    $$x^{(k+1)} = Tx^{(k)} + c$$
    (Hint: consider the splitting $A=D-L-U$).

\item[\quad] (b)
    For the particular case
    $$A= \begin{bmatrix} 2&1 \\ 1&2 \end{bmatrix}$$
    varify that the choice $\omega = 8/7$ gives the best rate of convergence.

\item[8.]
Given $x= (x_1, x_2, \dots, x_n)^T \in R^n$ ($T$ means transpose), define
$v = x + \hbox{sign} (x_1) \parallel x \parallel _2 e_1$, where
$e_1 = (1,0, \dots, 0)^T$. The Householder matrix (Householder
transformation) with $v$ (Householder vector) is given by
$$P=I - 2 \frac{vv^T}{v^Tv},$$
which is orthogonal and symmetric.

\item[\quad] (a)
Verify that $Px = -\hbox{sign} (x_1) \parallel x \parallel_2 e_1$.

\item[\quad] (b)
Describe how the Householder matrices can be used to construct an orthogonal
matrix $Q$ for a given matrix $A \in R^{n \times n}$ such that
$$A=QR,$$
where $R$ is upper tirangular.


\end{description}    
%\end{Large}
\end{document}














