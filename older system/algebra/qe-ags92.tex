% latex file
\def\hcorrection#1{\advance\hoffset by #1 }
\def\vcorrection#1{\advance\voffset by #1 }

\documentclass{article}
\usepackage{my,amsxtra,amssymb,amsthm}

\vcorrection{-1.0in}
\hcorrection{-0.8in}
\textwidth 6.0in
\textheight 9.0in
\def\R{\mathbb R}
\def\C{\mathbb C}
\def\N{\mathbb N}
\def\Z{\mathbb Z}
\def\Q{\mathbb Q}


\begin{document}
%\begin{Large}


\begin{center}\begin{LARGE}
{\bf Algebra Qualifying Exam}\\ 
{\bf Spring 1992}\\ \end{LARGE}
\end{center}
\vspace{0.1in}
\noindent\hrulefill\\
All rings are assumed to have a multiplicative identity, denoted 1. The
fields $\Q$, $\R$ and $\C$ are the fields of {\it rational, real} and
{\it complex} numbers, respectively.

\begin{description}

\item[1.]
Prove that no group of order 120 can be a simple group.

\item[2.]
Let $G$ be a group, and let $H$, $K$ be {\it solvable}- subgroups of $G$
with $K$ normal. Prove that $HK$ is a solvable subgroup of $G$.

\item[3.]
Let $G$ be a finite group of order greater than 3, and let $\cal C$ be a
conjugacy class of elements in $G$. If $|{\cal C}| = \frac{1}{3} |G|$, show
that every element of $\cal C$ is an element of order 3.

\item[4.]
Let $R$ be a unique factorization domain in which every prime ideal is
maximal. Prove that every prime ideal is principal. (In fact, it turns out
that {\it every} ideal is principal, but you are not asked to prove this.)

\item[5.]
Let $R$ be a ring. Prove that the following three conditions are equivalent
for the left $R$-module $M$.

\item[\quad] (i)
Any increasing chain of submodules
$$M_1 \subseteq M_2 \subseteq \dots,$$

\item[\quad] (ii)
Any submodule of $M$ is finitely generated.

\item[\quad] (iii)
Any family of submodules of $M$ has a maximal member with respect to inclusion.

\item[6.]
Let $F$ be a field and let $R$ be the ring
$$R= \left\{ \begin{bmatrix} a&0 \\ 0&b \end{bmatrix} | a,b \in F \right\}.$$

Define the left $R$-modules
$$M_1 = \left\{\begin{bmatrix} a \\ 0 \end{bmatrix} | a \in F \right\},
  M_2 = \left\{\begin{bmatrix} 0 \\ b \end{bmatrix} | b \in F \right\}.$$

(The module action is matrix multiplication.) Prove that $M_1$ and
$M_2$ are {\bf not} isomorphic as left $R$-modules.

\item[7.]
Let $V$ be a vector space of dimension $n$, and let
$1_V \neq T:V \to V$ be a linear transformation of $V$.
If im$(T-1_V) \subseteq \ker (T-1_V)$, compute the minimal and characteristic
polynomials of $T$ on $V$.

\item[8.]
Let $F_q$ be the finite field of $q$ elements, and let $K$ be an extension
of $F_q$, of degree $n$.

\item[\quad] (a)
Prove that the map $\tau_q : K \to K$, $\tau_q(x) = x^q$
is an automorphism of $K$, and that $F_q$ is precisely the subfield of fixed
elements of $\tau_q$.

\item[\quad] (b)
Compute $Gal(K/F_q)$.

\item[9.]
Let $f(x) = x^5 - 2 \in \Q[x]$.

\item[\quad] (a)
Construct a splitting field $K \supseteq \Q$ for $f(x)$ over $\Q$.

\item[\quad] (b)
If $G=Gal(K/\Q)$, prove that no noidentity element of $G$ can fix
two roots of $f(x)$.







\end{description}    
%\end{Large}
\end{document}














