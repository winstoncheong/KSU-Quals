% latex file
\def\hcorrection#1{\advance\hoffset by #1 }
\def\vcorrection#1{\advance\voffset by #1 }

\documentclass{article}
\usepackage{my,amsxtra,amssymb,amsthm}

\vcorrection{-1.0in}
\hcorrection{-0.8in}
\textwidth 6.0in
\textheight 9.0in

\def\R{\mathbb R}
\def\C{\mathbb C}
\def\N{\mathbb N}
\def\Z{\mathbb Z}
\def\Q{\mathbb Q}

\begin{document}
%\begin{Large}






\begin{center}\begin{LARGE}
{\bf Algebra Qualifying Exam}\\ 
{\bf Sping 1987}\\ \end{LARGE}
\end{center}
\vspace{0.1in}
\noindent\hrulefill\\

Do two problems from each section.

\centerline{{\bf Group Theory}}

\begin{description}
\item[1.]
Let $G$ be a non-abelian group of order $pq$, where $p$ and $q$ are primes
and $p < q$. Prove that $p|q-1$.

\item[2.]
Prove that there does not exist a simple group of order 112.

\item[3.]
Prove the Frattini lemma: if $G$ is a finite group, $K$ is a normal
subgroup of $G$, and $P$ is a Sylow $p$-subgroup of $K$, then $G = N_G(P)K$.

\item[4.]
Prove that any finite nilpotent group $G$ is the (internal) direct product
of its Sylow subgroups. (One possible approach: use the fact that
$N_G (H) \neq H$ for any proper subgroup $H$ of $G$.)

\centerline{{\bf Linear Algegra}}

\item[1.]
Let $V$ be a finite-dimensional ${\bf F}$-vector space and let
$T \in \hbox{End}_F (V)$. Assume that the minimal polynomial of $T$ is of the
form $f(x) g(x)$ where $f(x)$ and $g(x)$ are relatively prime polynomials
in ${\bf F} [x]$. Prove that $V = \ker f(T) \oplus \ker g(T)$
(internal direct sum).

\item[2.]
Let $V$ be an ${\bf F}$-vector space and let $W$ be a subspace of $V$. Let
$\widehat V$ and $\widehat W$ denote the dual spaces of $V$ and $W$,
respectively. Prove that
$$\widehat W \cong \widehat V / Ann(W)$$
where $Ann(W)$ is $\{f \in \widehat V | W \subseteq  \ker f\}$.

\item[3.]
Let $V$ be a vector space of dimension 7 over the rationals, and let
$T \in \hbox{End}_Q(V)$. Suppose that the characteristic polynomial of $T$
is $(x-1)^5(x-2)^2$ and that the minimal polynomial of $T$ is
$(x-1)^4(x-2)$. List the possibilities for the Jordan canonical form of
$T$, up to re-ordering of Jordan blocks.

\item[4.]
Let $V$ be a finite-dimensional vector space over an algebraically closed
field ${\bf F}$, and let $S$ and $T$ be two {\it commuting} members of
$\hbox{End}_{\bf F} (V)$. Show that $S$ and $T$ have a common eigenvector
(not necessarily for the same eigenvalue).

\centerline{{\bf Rings and Modules}}
(In these problems, rings are assumed to have a multiplicative identity
element ``1", and modules are assumed to be unital. That is, if $R$ is a ring
and $M$ is a (left) $R$-module, then $1 \cdot x=x$ for all $x \in M$).

\item[1.]
By definition, and $R$-module is irreducible if it is non-zero and has
no proper non-zero submodules. Let $M$ and $N$ be two irreducible $R$-modules.
Prove that either $Hom_R (M,N)=0$ or $M \cong N$. Show also that
$Hom_R(M,M)$ is a division ring.

\item[2.]
Show that if ${\bf F}$ is an infinite field and
$f \in {\bf F}[x_1, \dots, x_n] \neq 0$.

\item[3.]
Let $R$ be an integral domain and let $\rho$ be a non-zero prime ideal of $R$.
Show that $R_\rho$ has a unique maximal ideal (where $R_\rho$ denotes the
``localization" of $R$ at $\rho$).

\item[4.]
Show that a module over a ring $R$ is always a homomorphic image of a free
$R$-module.

\centerline{{\bf Fields and Galois Theory}}

\item[1.]
Let $E$ be a splitting field for the polynomial $x^3-5$ over ${\bf Q}$. Find
all of the subfields of $E$.

\item[2.]
Let $F_0$ be a field of order 4 (i.e., having precisely four elements). Let
$t$ be transcendenatal over $F_0$, and put $F = F_0 (t)$, the function
field in one variable over $F_0$. Finally, put $E=F(u)$ where $u^3=t$.

\item[\quad] (a)
Show that $E/F$ is normal and separable.

\item[\quad] (b)
Determine the galois group of the extension $E/F$.

\item[3.]
Let $K$ be an extension field of the rationals, of finite degree. Prove that
$K$ contains only a finite number of roots of unity.

\item[4.]
Let $E/F$ be an extension field and let $\alpha \in E$. Show that $\alpha$
is algebraic over $F$ if and only if $[F(\alpha) : F]$ is finite.






\end{description}    
%\end{Large}
\end{document}














