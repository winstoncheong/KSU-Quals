% latex file
\def\hcorrection#1{\advance\hoffset by #1 }
\def\vcorrection#1{\advance\voffset by #1 }

\documentclass{article}
\usepackage{my,amsxtra,amssymb,amsthm}

\vcorrection{-1.0in}
\hcorrection{-0.8in}
\textwidth 6.0in
\textheight 9.0in

\def\R{\mathbb R}
\def\C{\mathbb C}
\def\N{\mathbb N}
\def\Z{\mathbb Z}


\begin{document}
%\begin{Large}


\begin{center}\begin{LARGE}
{\bf Algebra Qualifying Exam}\\ 
{\bf Fall 1989}\\ \end{LARGE}
\end{center}
\vspace{0.1in}
\noindent\hrulefill\\
{\bf Group Theory}

\begin{description}
\item[1.]
Let $\phi : G \to H$ be a surjective homorphism of groups, and let
$K=\ker \phi$. If $H_1$ is a subgroup of $H$ show that there is a
{\it unique} subgroup $G_1$ of $G$ such that

\item[\quad] (i)
$K \leq G_1$,

\item[\quad] (ii)
$\phi (G_1) = H_1$.

\item[2.]
Let $G$ be a group of order 56. Show that either

\item[\quad] (i)
a 2-Sylow subgroup is normal, or

\item[\quad] (ii)
a 7-Sylow subgroup is normal.

(Extra credit: Give examples of groups $G_1, G_2$ of order 56 such that a
7-Sylow subgroup of $G_1$ is not normal and a 2-Sylow of $G_2$ is not normal.)

\item[3.]
Let $P$ be a finite $p$-group ($p$ is prime), and let $H$ be a proper
subgroup of $P$. Show that $N_p (H) \supsetneqq H$.

\item[4.]
Prove that no group can be written as the union of two proper subgroups.
Give an example of a group which is a union of three proper subgroups.

\item[5.]
Let $A$ be an abelian group with generators $a,b$ and relations $2a-b=0$,
$-a + 2b = 0$. Compute the structure of $A$.

\item[6.]
Let $G$ be the group with presentation $<a,b |a^2 = b^3>$. Show that $G$ is
infinite. (Hint: This is not hard at all! Let $G_0$ be the subgroup of
$GL(2, \pi) = 2 \times 2$ nonsingular matrices with integer entries,
generated by
$a_0 = \begin{bmatrix} 0&1 \\ -1&0 \end{bmatrix}$,
$b_0 = \begin{bmatrix} 0&1 \\ -1&0 \end{bmatrix}$.
Show that $a_0, b_0$ satisfy the given relation, and that $G_0$ is infinite.)

\end{description}

{\bf Rings and Modules}

\begin{description}
\item[1.]
Let $\phi : R_1 \to R_2$ be a homomorphism of rings.

\item[\quad] (a)
If $I_2$ is an ideal of $R_2$, show that $\phi^{-1} (I_2)$ is an ideal of
$R_1$.

\item[\quad] (b)
If $I_1$ is an ideal of $R_1$, show by example that $\phi(I_1)$ need not be
an ideal of $R_2$.

\item[2.]
Prove that ``Chinese  Remainder Theorem": If $n$ is a positive integer with
$n=ab$, $a$ and $b$ relatively prime, then there is an isomorphism of rings
$$\frac{\Z}{(n)} \cong \frac{\Z}{(a)} \times \frac{\Z}{(b)}.$$

\item[3.]
Let $R$ be a ring and let $M$ be a left $R$-module. Let
Ann$(M) = \{r \varepsilon R |rM = 0\}$ be the {\it annihilator} of $M$.

\item[\quad] (a)
Show that Ann$(M)$ is a 2-sided ideal of $R$

\item[\quad] (b)
If $M$ is irreducible, and if $R$ commutative, show that there is an
isomorphism of $R$-modules
$$\frac{R}{\hbox{Ann} (M)} \cong M$$

\item[4.]
Let $R$ be an integral domain such that every ideal of $R$ is free. Prove
that $R$ is a principal ideal domain.

\item[5.]
Let $R$ be a ring and let $M$ be a left $R$-module. Prove the so-called
{\it Noether isomorphism theorem}: if $M_1, M_2$ are $R$-submodules of
$M$ then
$$\frac{M_1 + M_2}{M_2} \cong \frac{M_1}{M_1 \cap M_2} .$$
(Hint: Map $M_1 \to \frac{M_1 + M_2}{M_2}$ in the more or less obvious
way. Is the map surjective? What is the kernel?)

\end{description}

{\bf Linear Algebra}

\begin{description}
\item[1.]
Let $F$ be a field, and let $V$ be a vector space over $F$.

\item[\quad] (a)
Define what it means for a subset $S \subseteq V$ to be a {\it basis}.

\item[\quad] (b)
Using Zorn's lemma, show that any vector space has a basis.

\item[2.]
Let $\{v_1, \dots, v_n\}$ be a basis for the vector space $V$ over $F$. If
$w \varepsilon V$ satisfies $w \notin < v_2, \dots, v_n>$ (where
$< >$ means $F$-span), show that $\{w, v_2, \dots, v_n\}$ is a basis.

\item[3.]
Let $T:V \to V$ be a linear transformation such that $T^2 = T$.
Prove that the subspaces $TV$ and $(I-T)V$ are $T$-invariant and that
$V=TV \oplus (I-T)V$.

\item[4.]
Give an example of a matrix $A$ with rational entries such that minimal
polymonial $=(x+1)^2 (x^2+1)^2 (x^4+ x^3 + x^2 + x+1)$, characteristic
polynomial $=(x+1)^3(x^2+1)^3(x^4+x^3+x^2+x+1)$.

\item[5.]
Let $T_1, T_2 : V \to V$ be linear transformations, where $V$ is a finite
dimensional vector space over an algebraically closed field. If
$T_1T_2 = T_2T_1$, prove that there exists a vector
$v \varepsilon V$ which is an eigenvector for both $T_1$ and $T_2$.

\end{description}

{\bf Fields and Galois Theory}

\begin{description}
\item[1.]
Let $F \subseteq K$ be fields and let $\alpha \varepsilon K$.

\item[\quad] (a)
State what it means for $\alpha$ to be {\it algebraic} over $F$.

\item[\quad] (b)
Prove that $\alpha$ is algebraic over $F$ if $F[\alpha]$ is a finite
dimension $F$-vector space.

\item[2.]
Let $F$ be a finite field, and let $F^\ast$ be the non-zero elements of $F$,
regarded as a multiplicative group. Show that $F^\ast$ is a cyclic group.
(Hint: If $e =$ exponent of $F^\ast$, how many roots in $F$ are ther to the
polynomial $x^e - 1$?)

\item[3.]
Let $\sqrt[3]{2}$ be a real cube root of 2, and let $\zeta$ be the complex
number $\zeta = \exp \left(\frac{2 \pi i}{3} \right)$. Let
$K_1 = \Phi \left[ \sqrt[3]{2} \right]$, $K_2 = \Phi[\zeta]$,
$K_3 = \Phi \left[\sqrt[3]{2}, \zeta \right]$. Prove that $K_1$ is not normal
over $\Phi$ but that $K_2, K_3$ are normal over $\Phi$.

\item[4.]
Let $F \subseteq K$ be a seperable normal extension of $F_1$ and let $G$ be
the Galois group of the extension. Let $H$ be a subgroup of $G$ and let
$L$= field of invariants of $H$, i.e.
$L = \{ \alpha \varepsilon K|h\alpha = \alpha
\hbox{\ for} \hbox{\ all\ } h \varepsilon H \}$. Without using the
fundamental theorem of Galois theory, prove that $L$ is normal over
$F$ if and only if $H$ is a normal subgroup of $G$.





\end{description}    
%\end{Large}
\end{document}














