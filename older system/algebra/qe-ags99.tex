%From dbski@math.ksu.edu Wed Feb  3 08:06:41 1999



% latex file
\def\hcorrection#1{\advance\hoffset by #1 }
\def\vcorrection#1{\advance\voffset by #1 }
\input amssym.def
\input amssym
\documentclass
{article}
\pagestyle{plain}

\def\nodivide{\not\kern -0.13em\vert \ }


\vcorrection{-1.0in}
\hcorrection{-0.8in}
\textwidth 6.0in
\textheight 9.0in
\begin{document}
\begin{Large}
\newcommand{\C}{{\Bbb C}}
\newcommand{\Q}{{\Bbb Q}}
\newcommand{\Gal}{\mbox{\rm Gal}}
\newcommand{\Aut}{\mbox{\rm Aut}}


\begin{center}\begin{LARGE}
{\bf  Algebra Qualifying Exam}\\ 
{\bf January 21, 1999}\\ \end{LARGE}
\end{center}
\vspace{0.1in}
\noindent\hrulefill\\
{\bf Instructions:} You are given 10 problems from which you are to do 8.
 Please indicate those  8 problems which you would like  to be graded 
by circling the problem numbers on the  problem sheet. 
{\bf Note:} All rings in this exam are associative and with 1 and 
all integral domains are commutative.
%$\Bbb Q $ and $ \Bbb C $ are
%the fields of  rational and complex  numbers respectively.

%$\Bbb Z $ and $ \Bbb Q $ are
%the sets of the integers and rational numbers respectively.

\vspace{0.2in}

\begin{enumerate}


\item Let $G$ be a finite abelian group. 
\begin{enumerate}
\item State what it means for $G$ to be an {\em elementary abelian} $p$-group,
where $p$ is a prime number.
\item If $G$ is an elementary abelian $p$-group,  explain fully in what sense can $G$
be regarded as an ${\Bbb F}_p$-vector space, where ${\Bbb F}_p$
is the field of $p$ elements.
\end{enumerate}

\item Let $G=\langle x,y\rangle$ be a finite group, where $x,y$ are
{\em involutions}. Prove that $G$ has a normal subgroup of index 2.
(Look at $H=\langle xy\rangle$.)

\item Let $G$ be a group acting on the set $\Omega $. Assume that 
$\omega\in \Omega$, set $H=\mbox{Stab}_G(\omega )$, and assume that
$K$ is a subgroup of $G$ acting transitively on $\Omega $. Prove that
$G=KH$. 

\item Let $R$ be a commutative ring and assume that $M, M_1,M_2,\ldots , 
M_r$ are maximal ideals of $R$ with $M_1M_2\cdots M_r\subseteq M$. 
Prove that for some $i$, $M_i=M$. 

\item Let $R=\{\frac{a}{b}\in {\Bbb Q}|\, 2\nodivide b\}$, a subring
of the rational number field ${\Bbb Q}$. Show that $R$ has a unique maximal
ideal, and find it.


\item Let $R$ be a ring and let $M$ be an $R$-module. Assume that $M_1,M_2
\subseteq M$ with $M=M_1\oplus M_2$. Prove or give a counterexample to the
assertion: {\sl If $N\subseteq M$ is a submodule, then  
$$N\, =\, N\cap M_1\oplus N\cap M_2.$$}

\item Let $\alpha =\root\of{2+\,\root\of 2}\in {\Bbb C}$. Given that
$m_{\alpha ,{\Bbb Q}}(x)=x^4-4x^2+2$, and that the roots of $f(x)=
m_{\alpha ,{\Bbb Q}}(x)$ are $\alpha=\alpha_1=\root\of{2+\,\root\of 2},
\alpha _2=-\root\of{2+\,\root\of 2}, \alpha_3=\root\of{2-\,\root\of 2},
\alpha_4=-\root\of{2-\,\root\of 2}$, answer the following:
\begin{enumerate}
\item Compute the degree of the splitting field ${\Bbb K}$
 over ${\Bbb Q}$ of $f(x)$.
\item Show that the Galois group $\mbox{Gal}({\Bbb K}/{\Bbb Q})$ is
cyclic.
\end{enumerate} 

\item Let ${\Bbb F}={\Bbb F}_q$ be the finite field of 
$q\ (=p^r)$ elements, where
$p$ is prime, and let ${\Bbb K}={\Bbb F}_{q^3}\supseteq {\Bbb F}$.
Say that elements $\alpha ,\beta\in {\Bbb K}$ are {\em equivalent} if 
they have the same minimimal polynomial over ${\Bbb F}$. Clearly
this is an equivalence relation on ${\Bbb K}$. Compute the number of 
equivalence classes in ${\Bbb K}$ as a function of $q$. (Hint: this is
{\em extremely} easy.)
         
\item Let $T:V\to V$ be a linear transformation on a finite dimensional
vector space over the field ${\Bbb F}$. Suppose that $T$ has the
following invariant factors:

$$1+x,\ x^2(1+x),\ x^2(1+x)(1+x+x^2).$$

Answer the following questions:
\begin{enumerate}
\item What is $\mbox{dim}_{\Bbb F}V$?
\item Is $T$ injective?
\item What is the minimal polynomial of $T$
\item Does $T$ have a Jordan canonical form over ${\Bbb F}$ with respect to
an appropriate basis of $V$? (If this depends on the field  
give an example of a field ${\Bbb F}$, for which the answer is ``yes,"
and find the Jordan canonical form.)
\end{enumerate} 


 
\item 
Let ${\Bbb F}$ be a field. If $V$ is a finite-dimensional
 ${\Bbb F}$-vector space and if $T:V\to V$ is a linear transformation, we
have the notion of {\em minimal polynomial} $m_T(x)\in {\Bbb F}[x]$ of $T$. Likewise, if
${\Bbb K}\supseteq {\Bbb F}$ is a finite field extension, and if $\alpha\in {\Bbb K}$,
then we also have the notion of {\em minimal polynomial} $m_\alpha (x)\in  {\Bbb F}[x]$
of the field elements $\alpha$. These notions of minimal polynomial share many similarities {\em except} that $m_\alpha (x)$ is {\em always} irreducible, whereas $m_T(x)$ {\em need not} be
irreducible. Prove this.

    
\end{enumerate}
\end{Large}
\end{document}
















