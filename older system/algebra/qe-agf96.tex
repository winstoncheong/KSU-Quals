

% latex file
\def\hcorrection#1{\advance\hoffset by #1 }
\def\vcorrection#1{\advance\voffset by #1 }

\input amssym.def   %defines Bbb and gothic (frak)
\input amssym       %defines Bbb and gothic (frak)
\newcommand{\divs}{\ \mbox{\rule[-4.1pt]{.4pt}{15pt}}\ }
\newcommand{\notdivs}{\hbox{\divs\kern-.59em\hbox{/}}\ }

%\documentstyle{article}
\documentclass{article}
\pagestyle{empty}

\setcounter{page}{1}

\vcorrection{-1.0in}
\hcorrection{-0.8in}
\textwidth 6.0in
\textheight 9.0in

\begin{document}
\begin{Large}

\begin{center}\begin{LARGE}
{\Bbb Algebra Qualifying Exam}\\

\end{LARGE}
{\Bbb September 26, 1996}
\end{center}
\vspace{0.1in}
\noindent\hrulefill\

\noindent {\Bbb Instructions:} You are given 10 problems from which you
are to do 8.
\noindent {\Bbb Note:} All rings are assumed to have a multiplicative
identity, denoted 1. The fields ${\Bbb Q}, {\Bbb R}$ and ${\Bbb C}$ are the
fields of {\em rational, real} and {\em complex} numbers, respectively.

\vspace{0.2in}
\begin{enumerate}

%\item Prove that the group $({\Bbb Q}^\times,\cdot )$ is not a cyclic group.

\item Suppose that $G/Z(G)$ is cyclic. Show that, in fact, $G$ is abelian.

%\item Suppose that $G$ is a finite group of order $p^2q$, where $p$ and
%$q$ are primes (not necessarily distinct). Show that $G'< G$.

\item Let $G$ be a group.
\begin{description}
\item{(i)} State what it means for $G$ to be a {\em solvable group}.
\item{(ii)} Let $G$ be a group, $K\triangleleft G$ be a normal subgroup
of $G$. Show that $G$ is solvable if and only if both $K$ and $G/K$ are
solvable groups.
\end{description}

%\item Let $A=Z_n$ be the cyclic group of order $n$. Prove that
%$\mbox{End}(A)\cong {\Bbb Z}/(n)$, where $\mbox{End}(A)$ is the
%ring of endomorphisms of $A$.

\item Let $R$ be a commutative ring. Recall that an ideal $I\subseteq R$ is
{\em finitely generated} if there exist elements $x_1,x_2,\ldots ,x_k\in I$
such that $I=(x_1,x_2,\ldots ,x_k)$ ($=\{\sum r_ix_i|\ r_1,r_2,\ldots ,r_k\in
R\}$). Recall next that $R$ satisfies the {\em ascending chain condition},
or is {\em Noetherian} if any sequence $I_0\subseteq I_1\subseteq\cdots
\subseteq I_n\subseteq\cdots $ 
of ideals eventually stabilizes, i.e., there exists
an integer $M$ such that $m\geq M$ implies that $I_m=I_M$.
Now prove that the following are equivalent for the commutative
ring $R$:
\begin{description}
\item{(i)} Every ideal $I\subseteq R$ is finitely generated.
\item{(ii)} $R$ satisfies the ascending chain condition.
\end{description}
\item Let $R=\{\frac{a}{b}\in {\Bbb Q}|\ 2\notdivs b\}$, a subring of
the rational number field ${\Bbb Q}$. Prove that $R$ has a unique
proper maximal ideal, viz., the one generated by the element $2\in R$.

%\item Give examples, as below.
%\begin{description}
%\item{(i)} Give an example of a ring $R$ and an {\em indecomposable}, but
%{\em not irreducible} $R$-module $M$.
%\item{(ii)} Give an example of a ring $R$ and a {\em projective}, but
%{\em not free} $R$-module $M$.
%\item{(iii)} Give an example of a ring $R$ and an {\em injective} $R$-module
%$M$.
%\end{description}

\item Let $R$ be a ring and let $M$ be an irreducible $R$-module. Prove
that $M\cong R/{\cal M}$, where ${\cal M}\subseteq R$ is a maximal left ideal.

%\item Let $T:V\to V$ be a linear transformation on the vector space $V$,
%over the field ${\Bbb F}$. If $m_T(x)=\sum_{i=0}^na_ix^i$, show that
%$T$ is invertible if and only if $a_0\not= 0$.

%\item  Let $T:V\to V$ be a linear transformation on the vector space $V$,
%over the field ${\Bbb F}$, with minimal polynomial $m_T(x)$. Prove that
%$\lambda\in {\Bbb F}$ is an eigenvalue of $T$ if and only if $\lambda$ is 
%a root of $m_T(x)$.

\item Let $V$ be a finite dimensional vector space with dual space $V^*$.
If $W\subseteq V$ is a subspace, set $\mbox{Ann}(W)=\{f\in V^*|\ f(w)=
0 \mbox{ for all }w\in W\}$, the {\em annihilator} of $W$ in $V^*$.
If $W_1,W_2$ are subspaces of $V$, show that $\mbox{Ann}(W_1+W_2)=
\mbox{Ann}(W_1)\cap \mbox{Ann}(W_2).$ 
%(It's also true that
%$\mbox{Ann}(W_1\cap W_2)=\mbox{Ann}(W_1)+\mbox{Ann}(W_2)$, but this is
%somewhat more difficult.)

\item Let $T:V\to V$ be a linear transformation on the vector space $V$,
over the field ${\Bbb F}$. Assume that $T$ has the following property:
whenever $W\subseteq V$ is a $T$-invariant subspace of $V$ then
there exists another $T$-invariant subspace $W'\subseteq V$ with the
property that $V=W\oplus W'$. Must $T$ be diagonalizable? Prove, or give
a counterexample.

\item Let ${\Bbb F}\subseteq {\Bbb K}$ be an algebraic extension of
fields. If $\alpha\in {\Bbb K}$, prove that the minimal polynomial
$m_{\alpha ,{\Bbb F}}(x)$ of $\alpha $ over ${\Bbb F}$ is an
{\em irreducible} polynomial.

%\item Let ${\Bbb K}\supseteq {\Bbb Q}$ be the splitting field of the
%irreducible polynomial $f(x)\in {\Bbb Q}[x]$. If the Galois group
%$A=\mbox{Gal}({\Bbb K}/{\Bbb Q})$ is abelian, prove that $\mbox{deg }f(x)=
%|A|.$ (Hint: $A$ acts transitively on the roots of $f(x)$.)

\item Let ${\Bbb K}={\Bbb Q}(\root\of{2+\root\of 2})$. Prove that
${\Bbb K}$ is a Galois extension of ${\Bbb Q}$. (Hint: show that
if $m(x)$ is the minimal polynomial of $\root\of{2+\root\of 2}$ over
${\Bbb Q}$, then $m(x)$ splits completely in ${\Bbb K}[x]$.

\item Let $p$ be a prime and let ${\Bbb F}={\Bbb F}_p$ be the finite field of
order $p$. Let $f(x)=x^2+x+1\in {\Bbb F}[x]$ and let ${\Bbb K}\supseteq
{\Bbb F}$ be the splitting field of $f(x)$ over ${\Bbb F}$.
Compute $[{\Bbb K}:{\Bbb F}] $ in the cases:
\begin{description}
\item{(i)} $p=2$,
\item{(ii)} $p=3.$
\end{description}
 


\end{enumerate}
\end{Large}
\end{document}



