
% latex file
\def\hcorrection#1{\advance\hoffset by #1 }
\def\vcorrection#1{\advance\voffset by #1 }
\input amssym.def
\input amssym
%\documentstyle{article}
\documentclass{article}
\pagestyle{plain}


\vcorrection{-1.0in}
\hcorrection{-0.8in}
\textwidth 6.0in
\textheight 9.0in
\begin{document}
\begin{Large}

\begin{center}\begin{LARGE}
{\bf  Algebra Qualifying Exam}\\ 
{\bf February 23, 1995}\\
\end{LARGE}
\end{center}
\vspace{0.1in}
\noindent\hrulefill\\
{\bf Instructions:} You are given 10 problems from which you are to do 8.
 Please indicate those  8 problems which you would like  to be graded 
by circling the problem numbers on the  problem sheet. 
{\bf Note:} All rings in this exam are associative and with 1 and 
all integral domains are commutative. $\Bbb Z $ and $ \Bbb Q $ are
the sets of the integers and rational numbers respectively.

\vspace{0.2in}

\begin{enumerate}
 \item  Let $ G $ be a finite group and $ p $ be the smallest prime divisor
 of $ | G | $. If $ H $ is a subgroup of $ G $ of index $ p $ in $ G $, 
show that $H$ is a normal subgroup.
\item Let $ p $ and $ q $ be  prime numbers. Show that any group of
order $ p^2 q $ is solvable. 

\item  Let $D=\Bbb Z[i]$, the ring of Gaussian integers. Compute
the order of the quotient ring $D/(1+2i)D$.

\item Let $f:R\to S$ be a homomorphism of rings, and let $I\subseteq R$
be an ideal. Is it true that $f(I)$ is an ideal of $S$? Prove, or
give a counterexample. What if $f$ is assumed to be surjective? 
 
\item Let $ B $ be a ring. An ideal $ I $ of $ R $ is 
called {\em nilpotent} if  
there exists a positive integer $n$ such that $ I^n=0 $ ($I^n=II\cdots I$).
Show that $IM= \{0\} $ for  any simple $ R $-module $ M $.

\item Let $ R $ be a ring and $ M $ an (left) $R$-module. An element $
m $ in $M$  is called a {\em torsion element} if $rm=0 $ for some   $0\neq r
\in R $. Let $ M_t $ be  the set of all
torsion elements in $ M $. Show that, if $ R $ is an integral domain,
then $ M_t $ is an $ R $-submodule and the quotient module $ M/M_t $
has no  torsion elements other than $ 0 $.

\clearpage

\item Let $ V $ be a finite dimensional vector space over an
algebraically closed field $ F $ and  $ T : V \rightarrow V $ be a
linear transformation. For each $ a \in F $, we define 
$ V_a=\{ v \in \; | \; (T-aI)^n v=0 \mbox{ for some positive}      $
\linebreak integer $n$\} , which is a $T$-invariant     
subspace of $V$. Here $ I$ is the identity linear transformation.
Show the  following:

(a). $ V_a \neq \{ 0 \} $ if and only if $ a $ is an eigenvalue of $ T
$.

(b). Let $ \Pi $ be the set of all eigenvalues of $ T $. Then $ V=
\oplus_{a \in \Pi} V_a $.

\item  Let $V $ be a finite dimensional vector space over a field $ F $ and 
$ A, B : V \rightarrow V $ be two commuting linear transformations. If
both $ A$ and $ B$ are  diagonalizable, then  there exists a basis of $ V $
such that  both  $A$ and $B$ have  diagonal matrices with
respect to this basis. 

\item Let $E$ be the splitting field $ f(x)=(x^3-2)(x^2+x+1)$ 
over $\Bbb Q$. Compute the Galois group $ \mbox{Gal}(E/\Bbb Q)$.

\item Let $E$ be the splitting field of $f(x)=x^5-2 $ over the field 
$ \Bbb F_5$, the field of 5 elements. Is $E$ a Galois extension 
over $\Bbb F_5$? Justify your
answer. If your answer is ``yes'',  compute the Galois group.  


\end{enumerate}
\end{Large}
\end{document}















