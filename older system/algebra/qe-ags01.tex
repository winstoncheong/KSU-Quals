
% latex file
\def\hcorrection#1{\advance\hoffset by #1 }
\def\vcorrection#1{\advance\voffset by #1 }
\input amssym.def
\input amssym
\documentclass
{article}
\pagestyle{plain}

\def\nodivide{\not\kern -0.13em\vert \ }


\vcorrection{-1.25in}
\hcorrection{-1in}
\textwidth 6.5in
\textheight 9.5in
\begin{document}
%\begin{Large}
\newcommand{\C}{{\Bbb C}}
\newcommand{\Q}{{\Bbb Q}}
\newcommand{\Z}{{\Bbb Z}}
\newcommand{\N}{{\Bbb N}}
\newcommand{\F}{{\Bbb F}}
\newcommand{\K}{{\Bbb K}}
\newcommand{\E}{{\Bbb E}}

\newcommand{\Gal}{{\rm Gal}}
\newcommand{\Aut}{{\rm Aut}}



\begin{center}\begin{LARGE}
{\bf  Algebra Qualifying Exam}\\ 
{\bf January 23, 2001}\\ \end{LARGE}
\end{center}
\vspace{0.1in}
\noindent\hrulefill\\
{\bf Instructions:} You are given 10 problems from which you are to do 8.
 Please indicate those  8 problems that you would like  to be graded 
by circling the problem numbers on the  problem sheet. Please understand
that one problem done completely correctly is worth quite a bit more
than two problems each half completed.
{\bf Note:} All rings in this exam are associative and with 1 and 
all integral domains are commutative.
%$\Bbb Q $ and $ \Bbb C $ are
%the fields of  rational and complex  numbers respectively.

%$\Bbb Z $ and $ \Bbb Q $ are
%the sets of the integers and rational numbers respectively.

\vspace{0.2in}

\begin{enumerate}

\item The group $G$ is called a $CA$-group if for every $e\not=x\in G,\
\mbox{C}_G(x)$ is abelian. Prove that if $G$ is a $CA$-group, then
\begin{enumerate}
\item the relation $x\sim y$ if and only if
$xy=yx$ is an equivalence relation on $G^\#$;
\item If $\cal C$ is an equivalence class in $G^\#$, then $H=
\{e\}\cup\cal C$ is a subgroup of $G$.
\end{enumerate}

\item Let $G$ be a group and let $M,N\triangleleft G$. If $G=MN$, prove
that $G/(M\cap N)\cong G/M\times G/N$.

\item Let $G=GL_2(p)$, $p$ prime, be the group of invertible $2\times
2$
matrices over the field $\F_p$.  Using the fact that $|G|=p(p-1)(p^2-1)$,
compute the number of Sylow $p$-subgroups of $G$.


\item Let ${\Z}[i]$ be the ring of Gaussian integers and let
$I\subseteq {\Z}[i]$ be an ideal. Assume that $I$ is invariant
under complex conjugation, i.e., $x\in I$ implies that
$\overline{x}\in I$. Prove that $I$ must be one of the types:
\begin{description}
\item{(i)} $I=a{\Z}[i]$, 
\item{(ii)} $I=ai{\Z}[i]$,  or
\item{(iii)} $I=a(1+i){\Z}[i]$,
\end{description}
where $a\in {\Z}$

\item Prove that no principal ideal in the polynomial ring ${\Z}[x]$
can be maximal.

\item Let $f:R\to R$ be a surjective homomorphism of the
noetherian domain $R$. Prove that $f$ is injective.

\item Let $V$ be an $n$-dimensional vector space over a field $\F$,
and let
$$V\ =\ V_0\supseteq V_1\supseteq\cdots\supseteq V_n\ =\ 0$$
be a chain of subspaces of $V$, with $\mbox{dim}(V_i/V_{i+1})=1$
for $i=0,1,\ldots ,n-1$.  Suppose that $T:V\to V$ is a linear transformation
satisfying $T(V_i)\subseteq V_{i+1}$ for all $i=0,1,\ldots ,n-1$.
Compute the characteristic polynomial of $T$.

\item Let $\C$ be the field of complex numbers, and let
$A\in\mbox{M}_3(\C)$ be the matrix
$$A\ =\ \left[\begin{array}{rrr}
              -1  &    1   &  1  \\
               2  &    0   & -1   \\
              -3  &   -3   & -2  \end{array}\right].$$
\begin{description}
\item{(a)} Compute the invariant factors of $A$.
\item{(b)} Compute the  Jordan cononical form of $A$.
\end{description}

\item Let $V$ be an $\F$-vector space and let $T:V\to V$
be a linear transformation. Assume that $T$ is {\em irreducible} in
that $V$ has no $T$-invariant subspaces. If we set
$$C_V(T)=\{\mbox{linear transformations }S:V\to V|\ ST=TS\},$$
prove that $C_V(T)=\F[T]$, {\em i.e.}, any linear transformation
on $V$ that commutes with $T$ is a polynomial in $T$. [Hint: Show that
$V$ is a $1$-dimensional vector space over the {\em field} $\F[T]$. How
does this help?]

\item Let $f(x)\in {\Q}[x]$ be an irreducible polynomial with
Galois group $G$ over the rational field ${\ Q}$. Assume that
every subgroup of $G$ is normal. Prove that ${\rm deg}\,f(x)=|G|.$

%\item Let ${\F}$ be a field. Prove that the groups
%$({\F},+)$ and $({\F}^\times,\times)$ cannot be
%isomorphic, where ${\F}^\times = {\F}-\{0\}$.



\end{enumerate}
%\end{Large}
\end{document}















