% latex file
\def\hcorrection#1{\advance\hoffset by #1 }
\def\vcorrection#1{\advance\voffset by #1 }

\documentclass{article}
\usepackage{my,amsxtra,amssymb,amsthm}

\vcorrection{-1.0in}
\hcorrection{-0.8in}
\textwidth 6.0in
\textheight 9.0in
\def\R{\mathbb R}
\def\C{\mathbb C}
\def\N{\mathbb N}
\def\Z{\mathbb Z}
\def\Q{\mathbb Q}

\begin{document}
%\begin{Large}


\begin{center}\begin{LARGE}
{\bf Algebra Qualifying Exam}\\ 
{\bf Spring 1994}\\ \end{LARGE}
\end{center}
\vspace{0.1in}
\noindent\hrulefill\\
All rings are assumed to have a multiplicative identity, denoted 1.
The fields $\Q$, $\R$ and $\C$ are the fields of
{\it rational, real} and {\it complex} numbers, respectively.

\begin{description}

\item[1.]
Let $G$ be a finite group and $p$ is a prime number. Define
$G(p) = \{g \in G| o(g) = p^n \hbox{\ for\ } \hbox{\ some\ } n\}$.

\item[\quad] (a)
Show that $G(p)$ is the union of all Sylow $p$-subgroups of $G$.

\item[\quad] (b)
Show that $G(p)$ is a subgroup if and only if $G$ has a normal Sylow
$p$-subgroup.

\item[2.]
Show that if $G$ is a finite $p$-group, then for any diviser $d$ of $|G|$,
$G$ has a normal subgroup of order $d$.

\item[3.]
Prove or disprove the following statements:

\item[\quad] (a)
An ideal $I$ of a commutative ring $R$ with 1 is maximal if and only if
$R/I$ is a field.

\item[\quad] (b)
An ideal $I$ of a ring $R$ with 1 is maximal if and only if $R/I$ is
a division ring.

\item[4.]
A commutative ring $R$ with 1 is called {\it local} if $R$ has only one
maximal ideal $m$. Show that in this case, the maximal ideal $m$ is
precisely the set of all non-units in $R$. Is it true in general that for
any commutative ring the set of all non-units is an ideal?

\item[5.]
Let $R$ be a ring with 1. An element $e \in R$ is called a central idempotent
if $e^2=e$ and $e$ is in the center of the ring $R$.

\item[\quad] (a)
Give an example of a ring $R$ having a central idempotent different from
0 and 1.

\item[\quad] (b)
Let $e \in R$ be a central idempotent show that for any unitary $R$-module
$M$, both $eM$ and $(1-e)M$ are $R$-submodules of $M$ and that
$M = eM \oplus (1-e)M$.

\item[6.]
Let $V$ be an $n$-dimensional vector space over a field $F$ and $T$:
$V \to V$ be a linear transformation. Set $P=\{x \in V | Tx = x\}$ to be
the subspace of $T$-fixed points and assume that $T(V) \subseteq P$.
Calculate the characteristic polynomial and minimal polynomial of $T$ in
terms of $n$ and $k = \dim \ker(T)$. Can $T$ be diagonalized?

\item[7.]
For $V$ a vector space over the field $F$, let $V^\ast$ denote the dual
space of $V$, that is, $V^\ast$ is the vector space
$Hom_F (V,F)$ of all linear transformations $\lambda : V \to F$.
If $V$ is $n$-dimensional with a basis
${\cal B} = \{x_1, x_2, \dots, x_n\}$, define elements
$\lambda_1, \dots, \lambda_n$ of $V^\ast$ by setting
$$\lambda_i \left( \sum^n_{j=1} a_jx_j \right) = a_i,$$
$1 \leq i \leq n, a_j \in F$, and put
${\cal B}^\ast = \{\lambda_1, \dots, \lambda_n\}$.

\item[\quad] (a)
Show that ${\cal B}^\ast$ is a basis of $V^\ast$.

\item[\quad] (b)
If $V$ is infinite dimensional with a basis $\{e_1, e_2, \dots, e_n, \dots\}$
and if the $\lambda_i$'s are defined similarly as above for $i= 1,2, \dots,$
prove or disprove the statement that $\{\lambda_1, \lambda_2, \dots\}$ is a
basis for $V^\ast$.

\item[8.]
Give an example of a normal field extension which is not Galois.

\item[9.]
Prove that any finite extension of degree $n$ over a finite field is
Galois. What is the Galois group?






\end{description}    
%\end{Large}
\end{document}














