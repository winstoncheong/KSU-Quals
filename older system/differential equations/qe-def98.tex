%From levkapit@math.ksu.edu Wed Sep  9 10:23:54 1998
%Date: Wed, 9 Sep 1998 10:09:15 -0500 (CDT)
%From: Lev Kapitanski <levkapit@math.ksu.edu>
%To: math@math.ksu.edu
%Cc: levkapit@math.ksu.edu
%Subject: qual

%Sheree,
 
%Here it is.

%Lev

%%%%%%%%%%%%%%%%%%%%%%%%%%%%%%%%%

%\pageno=1
\magnification \magstep1
\input amstex
\documentstyle{amsppt}
%\documentclass{amsppt}
\NoRunningHeads
%\NoPageNumbers
%\TagsOnRight
\baselineskip 15pt
%\hfuzz=10pt

\pagewidth{6.4 truein}
\pageheight{8.6 truein}
%\hcorrection{}
%\vcorrection{}


\topmatter
\title Differential Equations Qualifying Exam, Fall 1998
\endtitle
\author Department of Mathematics, Kansas State University 
\endauthor
\endtopmatter
%\document

{\bf 1)\/}\  Solve the Cauchy problem
$$
\cos(y) {\partial u\over \partial x} + {\partial u\over \partial y} 
+2 \tan (y) u -2 \tan (y)=0,
u(x,0)=h(x) 
$$
\bigskip

{\bf 2)\/} \ {\bf a)\/}\ Find a $\,2\pi$-periodic in $x$ solution of the 
equation
$$
u_t-u_{xx}=0
$$
that satisfies the initial condition 
$$
u(0,x)=\phi_N(x),
$$
where $\,N\ge 1\,$ is an integer and 
the function $\,\phi_N\,$ is defined on
$\,[-\pi,\,\pi]\,$  as follows
$$
\phi_N(x)= \cases N,\; \text{if}\; -1/2N<x<1/2N \\  
0,\qquad\text{otherwise},
\endcases  
$$
and extended periodically to the whole line. 
How many solutions are there? In what sense?
\medskip

{\bf b)\/}\ Let $\,u_N(t,x)\,$ denote the solution of the above problem such that   
$\,u_N(t,\cdot)\,$ is a strongly continuous function 
of $\,t\in [0,T]\,$ with values in $\, L^2([-\pi,\,\pi])$. 
Prove that such solution does
exist and is unique.  Is $\,u_N(t,x)\,$ is a bounded function of $\,x\,$ 
for every $\,t\ge 0$?
\medskip

{\bf c)\/}\ In what sense does there exist the limit $\,u_\infty=\lim_{N\to\infty} u_N\,$? 
Find $\,u_\infty$.
\vfill
\pagebreak

{\bf 3)\/}\ State and prove the maximum principle for the equation
$$
u_t-\Delta u + e^u = 0
$$
in $\,\Bbb R^n$. 

Use the maximum principle to state and prove a uniqueness theorem. 
\bigskip

{\bf 4)\/}\ Determine the type of the equation
$$
\sum_{i,j=1}^4 A_{ij}{\partial^2 u\over\partial x_i\partial x_j} 
+ \sum_{i=1}^4{\partial u\over\partial x_i} - u = 0,
$$
where $\,A_{ij}\,$ is the following matrix:
$$
\pmatrix 1 & 0 & 0 & 0 \\
         0 & 1 & 1 & 0 \\
         0 & 1 & 2 & 0 \\
         0 & 0 & 0 & -1 
\endpmatrix
$$

State and prove the finite domain of dependence
property  for this equation.
\bigskip

{\bf 5)\/}\ For the equation
$\,-\Delta u + u = f\,$ on the torus $\,\Bbb T^n$,  
prove the  estimate
$$
\sum_{j,k=1}^n\|u_{x_j x_k}\|_{L^2(\Bbb T^n)}\le 
C (\|f\|_{L^2(\Bbb T^n)}+\|u\|_{L^2(\Bbb T^n)})
$$
with a constant $\,C\,$ independent of $\,f$.
\bigskip

{\bf 6)\/}\ Find all distributional solutions of the equation 
$$
x y^\prime(x)=1.
$$
\bigskip

{\bf 7)\/} State and prove a finite dimensional version of 
Fredholm's Alternative.
\bigskip

{\bf 8)\/} Consider the Cauchy problem for the wave equation on $\,\Bbb R^n$:
$$
u_{tt}-\Delta u = 0, \quad u(0,x)=f(x),\;u_t(0,x)=0.
$$
In the 1-dimensional case ($n=1$) show that there exists a constant 
$\,C>0\,$ such that 
$$
\sup_t \|u(t,\cdot)\|_{L^1(\Bbb R^1)}\le C\,\|f\|_{L^1(\Bbb R^1)},
$$
for all $\,f\in L^1(\Bbb R^1)$.  

In the 3-dimensional case ($n=3$) prove that there is no constant such
that the above inequality holds. (Hint: consider spherically-symmetric 
solutions, use the change of variables $\,u(t,r)=r^\alpha v(t,r)\,$ 
with an appropriate $\,\alpha$ to obtain the 1-D wave equation for $v$, 
and choose $v(0,r)$ to have a support in an annulus.)



\end


%%%%%%%%%%%%%%%%%%%%%%%%%%%%%%%%%%%%%%%%%%%%%%%%%%%%%%
