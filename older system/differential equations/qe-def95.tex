% latex file
\def\hcorrection#1{\advance\hoffset by #1 }
\def\vcorrection#1{\advance\voffset by #1 }

\documentclass{article}
\usepackage{my,amsxtra,amssymb,amsthm}

\vcorrection{-1.0in}
\hcorrection{-0.8in}
\textwidth 6.0in
\textheight 9.0in
\begin{document}
%\begin{Large}






\begin{center}\begin{LARGE}
{\bf Differential Equations Qualifying Exam}\\ 
{\bf Fall 1995}\\ \end{LARGE}
\end{center}
\vspace{0.1in}
\noindent\hrulefill\\
\begin{description}

\item[1.] (i)
State a definition of fundamental solution of wave equations
$$u_{tt} (t,x) = a^2 \Delta u(t,x), \quad t > 0, x \in R^n.$$

\item[\quad] (ii)
Give the definitions of the Sobolev spaces
$W^{m,p} (\Omega), W_0^{m,p} (\Omega)$ where $p \geq 1$, and $\Omega$ is a
bounded domain with smooth boundary.

\item[2.]
Solve the initial value problem
$$\begin{cases}
        u_{tt} - u_{xx} = 1 \qquad \qquad t >0 , x \in R^1 \\
        u(0,x) = \begin{cases} 1, -1 \leq x \leq 1 \\ 0, |x|>1 \end{cases}\\
        u_t(0,x) = 0.
        \end{cases}$$

\item[3.] (i)
Formulate the Dirichlet principle.

\item[\quad] (ii)
Give a proof of the Dirichlet principle.

\item[4.]
A function is said to be a homogeneous function of degree $m\geq 0$, where
$m$ is an integer, if
$$f(\alpha x_1, \alpha x_2, \dots, \alpha x_n) = \alpha^m f(x_1, x_2, \dots,
   x_n)$$
holds true for any scalar $\alpha$. Show that $f(x_1, \dots, x_n)$ is
such a function if and only if
$$x_1 \frac{\partial f}{\partial x_1} + \dots + x_n
  \frac{\partial f}{\partial x_n} = m \cdot f.$$

\item[5.]
Prove that the only bounded harmonic functions in $R^n$ are constant functions.

\item[6.]
Solve the exterior Dirichlet problem in $R^3$:
$$\Delta u(x) = 0, \hbox{for\ } |x| > 1; \quad u(x) = c \hbox{\ for\ }
  |x| = 1.$$

\item[7.]
Find a solution to the heat equation
$$\begin{cases}
        u_t(t,x) = \Delta u(t,x) ; \quad &x = (x_1, x_2), 0\leq x^2_1 + x^2_2 <1 \\
        u(0,x) = 1 - r^2; \quad &r = \sqrt{x^2_1 + x^2_2} \\
        u(t,x)|_{r=1} = 0.
        \end{cases}$$





\end{description}    
%\end{Large}
\end{document}














