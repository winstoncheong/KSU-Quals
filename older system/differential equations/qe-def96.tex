%Date: Mon, 30 Sep 96 13:35:33 CDT
%From: Lev Kapitanski <levkapit@math.ksu.edu>
%Subject: PDE qual F96

%\pageno=1
%\magnification \magstep1
\input amstex
%\documentstyle{amsppt}
\documentstyle{amsppt}
\NoRunningHeads
%\NoPageNumbers
%\TagsOnRight
\baselineskip 15pt
%\hfuzz=10pt

\pagewidth{6.4 truein}
\pageheight{8.6 truein}
%\hcorrection{}
%\vcorrection{}

%\hskip 200bp {\bf Name:\/}%
\topmatter
\author Qualifying Examination in Differential Equations, Fall '96
\endauthor
\affil  Department of Mathematics,   
Kansas State University 
\endaffil
\endtopmatter
%\document
\centerline{(Try to get as many points as you can.)}
\vglue 0.5pc

{\bf \# 1.\/} \ 1) Find all the solutions of the equations:

(a)\ ({\it 1 point\/})\ \ $\,x\,{\partial u\over\partial y}\,+
\,y\,{\partial u\over\partial x}\,=\,0\,$;

(b)\ ({\it 2 points\/})\ \ $\,x\,{\partial u\over\partial y}\,+
\,y\,{\partial u\over\partial x}\,=\,1\,$;

(c)\ ({\it 2 points\/})\ \ $\,x\,{\partial u\over\partial y}\,+
\,y\,{\partial u\over\partial x}\,=\,u\,$;


2)\ ({\it 2 points\/})\ \  Solve the Cauchy problem:
$$
 x\,{\partial u\over\partial y}\,+
\,y\,{\partial u\over\partial x}\,=\,u\,, \qquad 
 u(x,y)\,=\,\sin(x-y)\qquad\text{on the line}\quad y=1-x\,.
$$
\vglue 1pc 
{\bf \# 2.\/} \ ({\it 4 points\/})\ What is a {\it well-posed\/} problem? 
Give an example.

\vglue 1pc 
{\bf \# 3.\/} \ ({\it 4 points\/})\ State a maximum principle for the 
solutions of the heat equation.
\vglue 1pc
{\bf \# 4.\/}\ (a)\ ({\it 2 points\/})\ Give an example of a function 
which is not differentialble in the classical sense but has a 
distributional (generalized) derivative in $\,L^2_{loc}$.

(b)\ ({\it 3 points\/})\ \ Give a definition of the Sobolev space $W^{m,p}(\Omega)$ (the same as $\,W^m_p(\Omega)$), where 
$\,1\le p\le +\infty$, $\,m\,$ is a positive integer, 
$\,\Omega\,$ is a bounded domain in $\,\Bbb R^n\,$ with smooth
boundary.
\vglue 1pc  
{\bf \# 5.\/}\  ({\it 5 points\/})\ \   Find  the solution of 
the initial boundary value problem:
$$\multline
\hfill {\partial u\over\partial t} - {\partial^2 u\over\partial x^2} - 
2\,{\partial u\over\partial x}\,=\,0\,,\qquad t>0,\quad 0<x<1\,,\qquad\hfill\\
\hfill u(t,0) = u(t,1) = 0,\qquad
\phantom{\qquad t>0,\quad 0<x<1\,,}\hfill\\
\hfill u(0,x) = e^{-x}\,\sin (11\pi x)\,.
\phantom{\qquad t>0,\quad 0<x<1\,,}\hfill
\endmultline
$$
\vglue 1pc

{\bf \# 6.\/}\ ({\it 5 points\/})\ \ Solve the following boundary 
value problem in the 
square $\,\Omega = \{ (x,y):\; 0<x,\,y<1\}$:
$$
- {\partial^2 u\over\partial x^2}- {\partial^2 u\over\partial y^2} 
- 2\,\pi^2\,u\,=\,\sin (\pi x)\,\sin (\pi y)\,,\;(x,y)\in\Omega ,\qquad
u\big |_{{}_{(x,y)\in\partial\Omega}}\,=\,0\,.
$$
\vglue 1pc
{\bf \# 7.\/}\ ({\it 5 points\/})\ \ Let $\,u(x)\,$ be the solution of 
the following boundary value problem in the disk 
$\,\Omega= \{ (x,y)\in\Bbb R^2: |x|^2+|y|^2 < 1\}$:
$$
- {\partial^2 u\over\partial x^2}- {\partial^2 u\over\partial y^2} 
+ u\,=\,0\,,\quad (x,y)\in\Omega ;\qquad 
u(x,y) \big |_{{}_{(x,y)\in\partial\Omega}}\,=\,{y\over\sqrt{|x|^2+|y|^2}}\,.
$$
Without solving the problem, show that $\,-1\le u(x,y)\le 1$\ \  for all 
$\,(x,y)\in\Omega$.
\vglue 1pc
{\bf \# 8.\/}\ ({\it 5 points\/})\ \ For a solution $\,\phi(t,x)\,$ of 
the Cauchy problem 
$$
{\partial^2 \phi\over\partial t^2} - 
{\partial^2 \phi\over\partial x^2}\,=\,0,\quad 
\phi(0,x)=f(x),\quad {\partial\phi\over\partial t} (0,x)= 0,
$$
show that if $\,f\in L^p(\Bbb R^1)$, for some $\,p$, $1\le p\le +\infty$, 
then,  $\,\phi(t,\cdot)\in L^p(\Bbb R^1)$, as well, 
and estimate $\,\sup_t \|\phi(t,\cdot)\|_{L^p}\,$ in terms of 
the $\,L^p$-norm of $\,f$.
\vglue 1pc 
{\bf \# 9.\/}\  ({\it 10 points\/})\ \  Consider the following hyperbolic
equation on $\,\Bbb R^1$:
$$
{\partial^2 \phi\over\partial t^2} - 
(1-\frac 12\cos^2 x){\partial^2 \phi\over\partial x^2}\,=\,0\,.
$$
Let $\,u\,$ and $\,v\,$ be the solutions of this equation corresponding 
to the initial conditions 
$$
u(0,x)=u_0(x),\quad {\partial u\over\partial t}(0,x)=u_1(x)\,,
$$
and
$$
v(0,x)=v_0(x),\quad {\partial v\over\partial t}(0,x)=v_1(x)\,.
$$
Show that if 
$$
u_0(x)=v_0(x),\qquad u_1(x)=v_1(x)\,,\qquad 
\text{for all $\,x\,$ in the interval}\;[-1,1]\,,
$$
then $\,u(t,x)=v(t,x)\,$ for all $\,(t,x)\,$ in the triangle 
$\,\{ 0\le t\le 1,\;|x|\le 1-t \}$. (You may assume that 
the solutions $\,u\,$ and $\,v$, and all the initial data, are as smooth 
as you wish.)
\vglue 1pc
%\pagebreak
 
{\bf \# 10.\/}\ ({\it 10 points\/})\ \  Let $\,\Omega\,$ be a bounded 
domain in 
$\,\Bbb R^n\,$ with smooth boundary. Consider the following 
initial boundary value problem:
$$
\align 
 u_t\,&=\,a^2\,\Delta\,u\quad t>0,\;x\in\Omega \\
 u(0,x)\,&=\,f(x),\quad x\in\Omega \\
 \tau\,{\partial u\over\partial \nu}\,+\,\sigma\,u \,&=\,0,\quad 
(t,x)\in [0,T]\times\partial\Omega\,,
\endalign
$$
where $\,a$, $\,\tau\,$ and $\,\sigma\,$ are positive constants. 
Show that 
$$
\int_\Omega |u(t,x)|^2\,dx 
$$
does not increase with $\,t$. Use this fact to prove the uniqueness 
of the (appropriately smooth) solutions. 


Show that the energy,
$$
E(t)\,=\,\int_\Omega \,|\nabla u|^2\,dx\,+\,
\int_{\partial\Omega} {\sigma\over \tau}\,|u|^2\,ds\,,
$$
does not increase with $\,t$.
 
 \end
