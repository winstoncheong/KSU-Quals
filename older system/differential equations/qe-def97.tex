% latex file
\def\hcorrection#1{\advance\hoffset by #1 }
\def\vcorrection#1{\advance\voffset by #1 }

\documentclass{article}
\usepackage{my,amsxtra,amssymb,amsthm}

\vcorrection{-1.0in}
\hcorrection{-0.8in}
\textwidth 6.0in
\textheight 9.0in
\begin{document}
%\begin{Large}






\begin{center}\begin{LARGE}
{\bf Partial Differential Equations Qualifying Exam}\\ 
{\bf Fall 1997}\\ \end{LARGE}
\end{center}
\vspace{0.1in}
\noindent\hrulefill\\

{\bf Part 1: The Laplace Equation}
\begin{description}

\item[1.1)]
Consider the following problem
$$-\Delta u (x) - 3u(x) = f(x), \quad x \in Q= \{x \in \Bbb R^3| \ 
  0 < x_j < 2\pi, j=1,2,3\}, \eqno{(1a)}$$
$$\frac{\partial u(x)}{\partial \nu} = 0, \quad x \in \partial Q,
  \eqno{(1b)}$$
where $\frac{\partial}{\partial \nu}$ is the normal derivative on the
boundary $\partial Q$ of the cube $Q$.

\item[\quad] a)
How would you define the solution of problem (1) for an $f \in L^2(Q)$?
State a theorem on the existence and uniqueness of a solution to (1) with
$f \in L^2 (Q)$.

\item[\quad] b)
Choose the function space for the solutions of (1) with $f$'s in an
appropriate subspace of $L^2(Q)$, and show that the solution depends on
continuously on $f$.

\item[\quad] c)
Solve problem (1) with $f(x) = \cos(x_1) \cdot \cos(x_2) \cdot \cos(x_3)$.

\item[\quad] d)
Explain the main difference between problem (1) and the following problem

$$-\Delta u(x) = f(x), \quad x \in Q= \{x \in \Bbb R^3 | \ 0< x_j< 2\pi,
  j =1,2,3\}, \eqno{(1'a)}$$
$$\frac{\partial u (x)}{\partial \nu} = 0, \quad x \in \partial Q.
  \eqno{(1'b)}$$

\item[1.2)]
Let $u$ be a $C^2$ function in a domain $\Omega \subset \Bbb R^n$.
Assume that $u$ is subharmonic in the sense that $\Delta u(x) \geq 0$ for
$x \in \Omega$.
Show that, for every $x \in \Omega$, and for every $r>0$ such that the closed
ball of radius $r$ centered at $x$ lies in $\Omega$, the following
inequality holds true
$$u(x) \leq \frac{1}{|\Bbb S^{n-1}|} \int_{|\xi|=1}
  u(x+r \xi) d_\xi \sum, \eqno{(2)}$$
where $|\Bbb S^{n-1}|$ is the area of the unit sphere $\Bbb S^{n-1}$ in
$\Bbb R^n$ and $d_\xi \sum$ is the natural measure on
$|\Bbb S^{n-1}|$.

\item[1.3)]
Let $\Omega$ be a bounded domain in $\Bbb R^n$ with smooth boundary
$\partial \Omega$. Show that if $u \in C^2(\Omega) \cap C(\overline \Omega)$
and $u$ is a solution of the problem
$$\Delta u(x) = u(x)^3, \quad x \in \Omega, \eqno{(3a)}$$
$$u|_{x \in \partial \Omega} = 0 \eqno{(3b)}$$
then $u(x) \equiv 0$ in $\Omega$.

\item[1.4)]
Give an example of a function $f$ in the ball
$B= \{x \in \Bbb R^n|\  |x| < 1\}$ such that
$f \in W^1_2(B)$, but $f \notin W^2_2(B)$.

\end{description}

{\bf Part 2: The Heat Equation}
\begin{description}

\item[2.1)]
Consider the following problem (the Cauchy problem for the heat equation
on the torus):
$$\frac{\partial}{\partial t} u(t,x) -\Delta u(t,x) = 0, \quad t>0,
  x \in \Bbb R^n, \eqno{(4a)}$$
$$u(0,x) = f(x), x \in \Bbb R^n, \eqno{(4b)}$$
$$u(t,x_1, \dots, x_j + 2 \pi, \dots, x_n) = u(t,x_1, \dots, x_j, \dots, x_n),
  \quad j =1, \dots, n, t>0, \eqno{(4c)}$$
where $f$ is assumed to be $2 \pi$-periodic in each of the variables.

\item[\quad] a)
Use Fourier series to find the solution $u$ of (4) in terms of the Fourier
coefficients of $f$.

\item[\quad] b)
Assume that $f$ belongs to the $2\pi$-periodic Sobolev space
$\widetilde W^1_2$. The
space $\widetilde W^1_2$ is defined as the closure of the set of
$C^\infty$-smooth
functions $2\pi$-periodic in each of the variables, in the norm
$$\parallel f \parallel = \left(\int_Q |f(x)|^2 +
  |\nabla f(x)|^2 dx \right)^{\frac{1}{2}},$$
where $Q=\{ y \in \Bbb R^n | \ 0 < y_j < 2\pi\}$
(an elementary cell). Express the fact $f \in \widetilde W^1_2$ in terms of
the Fourier coefficients of $f$.

\item[\quad] c)
Show that if $f \in \widetilde W^1_2$, then the formula you obtained for
$u(t,x)$ defines a function with the following properties:

\item[\quad] \quad 1)
$u(t, \cdot) \in \widetilde W^1_2$ for all $t>0$;

\item[\quad] \quad 2)
the mapping $t \mapsto u(t, \cdot) \in \widetilde W^1_2$
from $[0, +\infty)$ into $\widetilde W^1_2$ is continuous, and
$u(t, \cdot) \to f(\cdot)$ in $\widetilde W^1_2$ as $t \searrow 0$;

\item[\quad] \quad 3)
$u(t,x)$ satisfies (7a) in the distributional sense.

\item[2.2] a)
In the case $n=1$, find the solution $u(t,x)$ of (4) corresponding to the
initial condition
$$f(x) = x-\pi, \hbox{for\ } 0 \leq x < 2\pi, \hbox{and\ }
  \hbox{extended\ } \hbox{periodically\ } \hbox{to\ } \Bbb R^1.$$

You may leave the solution in the form of a Fourier series.

\item[\quad] b)
Show that $u(t,x)$ has the following properties:

\item[\quad] \quad 1)
$u(t, \cdot) \in L^2 (0, 2\pi)$ for all $t>0$;

\item[\quad] \quad 2)
the mapping $t \mapsto u(t, \cdot) \in L^2 (0, 2\pi)$ from $[0, +\infty)$
into $L^2(0, 2\pi)$ is continuous, and $u(t,\cdot) \to f(\cdot)$ in
$L^2(0,2\pi)$ as $t \searrow 0$;

\item[2.3] a)
Find a formula for the solution of the initial value problem:
$$\frac{\partial}{\partial t} u(t,x) = \Delta u(t,x) - u(t,x), \quad
  t>0, x \in \Bbb R^n, \eqno{(5a)}$$
$$u(0,x) = g(x), \eqno{(5b)}$$
where $g$ is continuous and bounded.

\item[\quad] b)
Is the solution bounded?

\item[\quad] c)
Is it the only bounded solution?

\end{description}

{\bf Part 3: The Wave Equation}

\begin{description}
\item[3.1]
Consider the Cauchy problem
$$\frac{\partial^2}{\partial t^2} u(t,x) - \Delta u(t,x) = 0, \quad
  -\infty < t< +\infty, \quad x \in \Bbb R^n, \eqno{(6a)}$$
$$u(0,x) = 0, \quad x \in \Bbb R^n, \eqno{(6b)}$$
$$\frac{\partial}{\partial t} u(t,x) |_{t=0} = f(x),
  \quad x \in \Bbb R^n. \eqno{(6c)}$$

Let $f$ be a function with the following properties:

$\bullet$ $f \in C^\infty (\Bbb R^n \backslash \{0\})$;

$\bullet$ $|f(x)| \leq C_0 |x|^\alpha$, for some $\alpha > -\frac{n}{2}$,
   when $x$ is small: $0 < |x| \leq \varepsilon$, some $\varepsilon >0$;

$\bullet$ $|f(x)| \leq C_\infty |x|^{-\beta}$, for some $\beta > \frac{n}{2}$,
   when $x$ is large: $|x| >R$, some $R>0$.

What can you say about the regularity properties of the solution
$u(t,x)$ of (6)?

\item[3.2]
Consider the problem
$$\frac{\partial^2}{\partial t^2} u(t,x) -\frac{\partial^2}{\partial x^2}
  u(t,x) = h(t,x), \quad -\infty < t +\infty, \quad -\infty < x < +\infty,
  \eqno{(7a)}$$
$$u(0,x) = g(x), \quad x \in \Bbb R^1, \eqno{(7b)}$$
$$\frac{\partial}{\partial t} u(t,x) |_{t=0} = f(x), \quad x \in \Bbb R^1.
  \eqno{(7c)}$$

Assuming that $f, g, h$, and $U$ are sufficiently smooth, integrate the
equation (7a) over the isoceles triangle (``backward light cone")
$C_{t_0, x_0} = \{(t,x) |0<t< t_0, |x-y| < t_0 -t\}$ with the vertices
$P=(t = t_0, x = x_0), Q=(t=0, x+x_0 + t_0)$, and $R= (t=0, x=x_0 - t_0)$.
In order to simplify the left side, use integration by parts to get
integrals over the sides of $C_{t_0, x_0}$, and then evaluate the
integrals over the sides $RP$ and $QP$. After this you should be able to
obtain an expression for $u(t_0, x_0)$ in terms of $f, g$, and $h$.

Use the expression you obtained to explain the Huygens principle.

\item[3.3]
$$\frac{\partial^2}{\partial t^2} u(t,x) - \Delta u(t,x) + m^2u(t,x) =
h(t,x), \quad -\infty < t< +\infty, \quad x \in \Bbb R^n, \eqno{(8a)}$$
$$u(0,x) = g(x), \quad x \in \Bbb R^n, \eqno{(8b)}$$
$$ \frac{\partial}{\partial t} u(t,x) |_{t=0} = f(x), \quad
   x \in \Bbb R^n, \eqno{(8c)}$$

where $m$ is a positive constant.

\item[\quad] a)
If the initial data $f$ and $g$ are chosen from the function spaces
$H^1(\Bbb R^n) = W^1_2(\Bbb R^n)$ and $L^2(\Bbb R^n)$, respectively,
then what is the ``right" function space for $h$ in order that problem
(8) be well-posed?

\item[\quad] b)
Write down the appropriate energy estimate and show how this estimate allows
to prove the uniqueness of solution of (8). Derive this estimate for the
solution of (8) under the assumptions that the initial data $f$ and $g$,
the external force $h$, and the solution $u$ are all sufficiently smooth
and decrease fast at the spatial infinity.



\end{description}    
%\end{Large}
\end{document}














