% latex file
\def\hcorrection#1{\advance\hoffset by #1 }
\def\vcorrection#1{\advance\voffset by #1 }

\documentclass{article}
\usepackage{my,amsxtra,amssymb,amsthm}

\vcorrection{-1.0in}
\hcorrection{-0.8in}
\textwidth 6.0in
\textheight 9.0in
\begin{document}
%\begin{Large}






\begin{center}\begin{LARGE}
{\bf Partial Differential Equations Qualifying Exam}\\ 
{\bf Spring 1992}\\ \end{LARGE}
\end{center}
\vspace{0.1in}
\noindent\hrulefill\\

\begin{description}
\item[1.]
Let $\Omega$ be a bounded smooth domain in $\Bbb R^2$ and
$f \in C^1(\partial \Omega)$.

\item[\quad] (i)
Is the following Neumann problem well posed? Why?
$$\begin{cases}
        \Delta u = 0 \\
        \frac{\partial u }{\partial n} |\partial \Omega = f
        \end{cases}$$

\item[\quad] (ii)
Find a necessary condition for this problem to have a solution and prove
your result.

\item[2.]
Suppose $u(x,t), x \in \Bbb R^1, t>0$ is a solution of
$\frac{\partial^2 u}{\partial x^2} - \frac{\partial^2 u}{\partial t^2} =0$.
Suppose that on
$I=[a,b] \subseteq \Bbb R, u(x,0) = \frac{\partial u}{\partial t} (x,0)=0$.
Show that $u \equiv 0$ in
$\Omega = \left\{(x,t)= 0 \leq t \leq \frac{a+b}{2}, \left|x- \frac{a+b}{2}
  \right| \leq \frac{a+b}{2} - t \right\}.$
HINT: Consider the energy integral
$E(t) = \frac{1}{2} \int_{B_t} (u_t)^2 + |\nabla u|^2 dx$, where
$$B_t = \left\{x : \left| x- \frac{a+b}{2} \right|
  \leq \frac{a+b}{2} - t \right\}.$$

\item[3.]
Consider the half disc $\Omega$ in $\Bbb R^2$ as in Fig. 1 below. Find a
solution $u(x,y)$ of the Dirichlet problem

\begin{tabular}{ll}
\rule{4in}{0in} & \\
$\begin{cases}
        \Delta u &= 0 \hbox{\ in\ } \Omega \\
        u(x,0) &= 0, -1 \leq x \leq 1 \\
        u|_{\partial \Omega - [-1,1]} &= 1.
        \end{cases}.$
&
\begin{small}
\setlength{\unitlength}{.0125in}
\begin{picture}(-100,0)%(-250,0)
\put(-50,0){\line(1,0){100}}
%\put(-50,0){\oval(50,100)[t]}
\qbezier[500](-50,0)(0,75)(50,0)
\put(-50,-10){$-1$}
\put(45,-10){$1$}
\put(-15,-15){\begin{normalsize}Fig.1\end{normalsize}}
\put(-5,23){$\Omega$}
\put(38,20){$\sigma,\Omega$}
\end{picture}
\end{small}
\end{tabular}


\item[4.]
Suppose $u$ is a continuous function on a domain $\Omega \subset \Bbb R^n$
which satisfies the mean value property, that is, whenever $B \subset \Omega$
is a ball centered at $x$ then
$u(x) = \frac{1}{|\partial B|} \int_{\partial B} u(y) dy$. Show that $u(x)$
is a harmonic function and is also $C^\infty$ on $\Omega$.

\item[5.]
Let $f$ be a $C^2$ function and $\phi$ be a $C^1$ function on $R^1$. Consider
the conservation law of nonlinear flow

$\begin{cases}
        u_t + (f(u))_x &=0 \\
        u(0,x) &= \phi (x),
        \end{cases}$

where $u=u(t,x)$ is the density of this flow. Show that if $f$ is convex and
$\phi$ is decreasing then this flow must undergo a blow-up at some time
$t_0 >0$.

\item[6.]
Let $\Omega$ be a smooth bounded domain in $\Bbb R^n$. Let
$f \in C(\partial \Omega)$ be fixed. Consider the functional
$Du = \int_\Omega |\nabla u|^2 dx$ where $u|_{\partial \Omega} = f$.
Show that $Du$ achieves a minimum when $u$ is harmonic.

\item[7.]
Suppose $\Delta u + u=0$ in $\Bbb R^3$ and
$|u(x)_{|\partial B_R}| \leq \frac{C}{1+R^2}$ for some constant $c>0$,
where $B_R = \{x \in \Bbb R^3 : \parallel x \parallel = R\}$. Does
$u \equiv 0$? Give your argument.

\item[8.] (i)
Give an example that a function is weakly differentiable but not
differentiable Give also the weak derivatives for your example.

\item[\quad] (ii)
Let $\Omega$ be an open bounded set with smooth boundary in $\Bbb R^3$. If
a function $u \in W^{2,2}_0 (\Omega)$, is $u(x)$ classically differentiable?
Why?

\item[9.]
Let $H$ be a real Hilbert space and $A$ a densely defined linear selfadjoint
operator on $H$. Assume
$\langle Au, u \rangle \geq c \parallel u \parallel^2, \forall u \in H$,
where $c>0$ is a constant. Let $f \in H$ be given.

\item[\quad] (i)
Show that if $u_0 \in H$ minimizes the functional
$F(u) = \langle Au, u \rangle - 2\langle f,u \rangle$, then
$Au_0 = f$.

\item[\quad] (ii)
In the case that $H = L^2(\Omega), A= -\Delta$ under the homogeneous
Dirichlet boundary condition. Specify the constant $c$ above.





\end{description}    
%\end{Large}
\end{document}














