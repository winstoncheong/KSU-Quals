% latex file
\def\hcorrection#1{\advance\hoffset by #1 }
\def\vcorrection#1{\advance\voffset by #1 }

\documentclass{article}
\usepackage{my,amsxtra,amssymb,amsthm}

\vcorrection{-1.0in}
\hcorrection{-0.8in}
\textwidth 6.0in
\textheight 9.0in
\begin{document}
%\begin{Large}






\begin{center}\begin{LARGE}
{\bf Partial Differential Equations Qualifying Exam}\\ 
{\bf Spring 1996}\\ \end{LARGE}
\end{center}
\vspace{0.1in}
\noindent\hrulefill\\

The bundle \#1: The Laplace Equation

\begin{description}

\item[1.1]
Consider the following problem:
$$-\Delta u(x) = f(x), \quad x \in B_R(0) = \{|x| < R\}, \eqno{(1a)}$$
$$u(x) = 0, \quad |x| = R, \eqno{(1b)}$$
where
$x=(x_1, \dots, x_n), n \geq 1, |x| = \sqrt{x^2_1 + \dots + x^2_n}, R>0$.

How would you define the solution of problem (1) for $f \in L^2(B_R(0))$?
State a theorem on the existence and uniqueness of solution to (1) with
$f \in L^2(B_R (0))$.

Choose the appropriate function space for the solutions of (1) with $f$'s
in $L^2(B_R(0))$, and show that the solution depends continuously on $f$.

\item[1.2]
The Poisson formula,
$$u(x) = \frac{1}{|\Bbb S^{n-1}|} \cdot \frac{R^2 -|x|^2}{R} \cdot
  \int_{|y| = R} \frac{\varphi (y)}{|x-y|^n} dS_y, \eqno{(2)}$$
where $|\Bbb S^{n-1}|$ is the area of the unit sphere $S^{n-1}$ in
$\Bbb R^n$, gives the solution of the following Dirichlet problem
in the ball $B_R(0)$,
$$-\Delta u(x) = 0, \quad x \in B_R (0) = \{|x| < R\}, \eqno{(3a)}$$
$$u(y) = \varphi (y), \quad |y| = R. \eqno{(3b)}$$
Use (2) to prove the following result ({\it Harnack's inequality}):
  
{\it If $u$ is a nonnegative harmonic function in the ball
$B_R(0)$, and $u \in C \overline{(B_R(0))} \cap C^2(B_R(0))$, then}
$$R^{n-2} \cdot \frac{R-|x|}{(R+|x|)^{n-1}} \cdot u(0) \leq u(x) \leq
  R^{n-2} \cdot \frac{R+|x|}{(R-|x|)^{n-1}} \cdot u(0),
  \eqno{(4)}$$
{\it for any $x \in B_R(0)$.}

Use (4) to prove the Liouville theorem:

{\it If the function $u$ is harmonic in $\Bbb R^n$ and bounded from below,
then $u \equiv const$.}

\item[1.3]
Let $\Gamma_\alpha$, $0 < \alpha < 2\pi$, be the sector on the plane defined
in the polar coordinates $(r, \theta)$ by the equation
$$\Gamma_\alpha = \{(r, \theta) : r > 0, 0 < \theta < \alpha\}.$$
Denote by $\partial \Gamma_\alpha$ the boundary of $\Gamma_\alpha$ (formed
by the two rays, $\{r \geq 0, \theta = 0 \}$ and
$\{r \geq 0, \theta = \alpha\}$).

Use the separation of variables method to find nontrivial solutions $u$ of the
Dirichlet problem
$$-\Delta u(x) = 0, \quad x \in \Gamma_\alpha, \eqno{(5a)}$$
$$u(x) = 0, \quad x \in \partial \Gamma_\alpha. \eqno{(5b)}$$

In the case $\pi < \alpha < 2 \pi$, choose the solution $u$ with the
properties:
$$u \in W^1_2 (\Gamma_{\alpha, R}), \hbox{\ but\ } u \notin W^2_2
  (\Gamma_{\alpha, R}). \eqno{(6)}$$
for any $R>0$, where $\Gamma_{\alpha, R}$ is the intersection of
$\Gamma_\alpha$ with the disc $\{0 \leq r < R\}$, and $W^1_2$, $W^2_2$
are the standard Sobolev spaces (sometimes the notation $W^{1,2}$,
$W^{2,2}$ is used for these spaces).

\end{description}

The bundle \#2: The Heat Equation

\begin{description}

\item[2.1]
Consider the Cauchy problem:
$$\frac{\partial}{\partial t} u(t,x) - \Delta u(t,x) = 0, \quad t>0,
  x \in \Bbb R^n, \eqno{(7a)}$$
$$u(0,x) = f(x). \eqno{(7b)}$$

Assuming that $f(\cdot)$ is smooth, rapidly decreasing function, use the
Fourier transform in $x$ to find the solution $u$ of (7).

(You may need the equality $\int^{+\infty}_{-\infty} e^{-s^2} ds =\sqrt{\pi}$.)

Show that if $f \in L^2 (\Bbb R^n)$, then the formula you obtained for
$u(t,x)$ defines a function with the following properties:

\item[\quad] a)
$u(t, \cdot) \in L^2 (\Bbb R^n)$ for all $t>0$;

\item[\quad] b)
the mapping $t \mapsto u(t, \cdot) \in L^2(\Bbb R^n)$ from $[0, +\infty)$
into $L^2(\Bbb R^n)$ is continuous, and $u(t, \cdot) \to f(\cdot)$ in
$L^2(\Bbb R^n)$ as $t \searrow 0$;

\item[\quad] c)
$u(t,x)$ satifies (7a) in the distributional sense.

\item[2.2]
In the case $n=1$, find the solution of (7) corresponding to the initial
condition
$$f(x) = e^{ax^2}, $$
where $a = const >0$. Find the life span of this solution (the life span
is the largest time interval where the solution is finite).

\item[2.3]
Show that if $u(0,x) = f(x)$ is an even (odd) function of $x$, then, for
every $t>0$, the solution $u(t,x)$ of (7) is an even(odd) function of $x$
as well.

Use this to solve the following initial boundary-value problem on the half-
line:
$$\frac{\partial}{\partial t} u(t,x) - \frac{\partial^2}{\partial x^2}
  u(t,x) = 0, \quad t>0, 0<x< +\infty, \eqno{(8a)}$$
$$u(0,x) = f(x) \equiv 1, 0<x< +\infty, \eqno{(8b)}$$
$$u(t,0) = 0, t>0. \eqno{(8c)}$$

\end{description}

The bundle \#3: The Wave Equation

\begin{description}
\item[3.1]
Consider the Cauchy problem
$$\frac{\partial^2}{\partial t^2} u(t,x) - \Delta u(t,x) = 0, \quad
  -\infty < t < +\infty, \quad x \in \Bbb R^n, \eqno{(9a)}$$
$$u(0,x) = 0, \quad x \in \Bbb R^n, \eqno{(9b)}$$
$$\frac{\partial}{\partial t} u(t,x) |_{t=0} = f(x), \quad x \in \Bbb R^n.
  \eqno{(9c)}$$
Assume that you know the operator $U(t)$ that solves this problem, i.e.,
$$U(t) : f(\cdot) \mapsto (U(t) f)(\cdot) = u(t, \cdot), \eqno{(10)}$$
where $u(t,)$ is the solution of (9).

Use this operator to express the solution of the more general problem,
$$\frac{\partial^2}{\partial t^2} u(t,x) - \Delta u(t,x) = h(t,x), \quad
  -\infty < t < +\infty, \quad x \in \Bbb R^n, \eqno{(11a)}$$
$$u(0,x) = g(x), \quad x \in \Bbb R^n, \eqno{(11b)}$$
$$\frac{\partial}{\partial t} u(t,x) |_{t=0} = f(x), \quad
  x \in \Bbb R^n, \eqno{(11c)}$$
in terms of the data: $f, g$, and $h$.

Hint: Looking at the Fourier transformation of the solution of (11) may
help you.

\item[3.2]
Consider the problem (11). If the initial data $f$ and $g$ are chosen from
the function spaces $H^1(\Bbb R^n) = W^1_2 (\Bbb R^n)$ and $L^2(\Bbb R^n)$,
respectively, then what is the ``right" function space for $h$ in order
that problem (11) be well-posed?

Write down the appropriate energy estimate that allows to prove the
uniqueness of the solution for (11). Derive this estimate for the solution
of (11) under the assumptions that the initial data $f$ and $g$, the
external force $h$, and the solution $u$ are all sufficiently smooth and
decrease fast at the spatial infinity.

\item[3.3]
What is the Huygens principle for the wave equation?

Solve problem (9) explicitly in the case $n=1$ and
$$f(x) = \begin{cases}
                1, \hbox{if\ } |x| \leq 1, \\
                0, \hbox{if\ } |x| > 1.
                \end{cases}$$
Explain how your solution illustrates the Huygens principle.

\item[3.4]
Use the method of separation of variables to solve the following initial
boundary-value problem:
$$\frac{\partial^2}{\partial t^2} u(t,x) - \frac{\partial^2}{\partial x^2}
  u(t,x) = 0, \quad t>0, \quad 0 < x< 1, \eqno{(12a)}$$
$$u(0,x) = \sin(\pi x), \quad 0 \leq x \leq 1, \eqno{(12b)}$$
$$\frac{\partial}{\partial t} u(t,x) |_{t=0} \equiv 1, \quad
  0 \leq x \leq 1, \eqno{(12c)}$$
$$u(t,0) = u(t,1) = 0, \quad t \geq 0. \eqno{(12d)}$$



\end{description}    
%\end{Large}
\end{document}














