% latex209 file
%Real Analysis - Bennett & Moore

\def\hcorrection#1{\advance\hoffset by #1 }
\def\vcorrection#1{\advance\voffset by #1 }


%\input prepictex.tex
%\input pictex.tex
%\input postpictex.tex

%\documentstyle[bbb]{report}
\documentclass[bbb]{report}
\usepackage{amsxtra,amssymb,amsthm,amsmath,latexsym}

\vcorrection{-.5in}
\hcorrection{-1.25in}

\topmargin  0in
%\textwidth 7in
%\textheight 10truein
\textheight 9truein
\textwidth 6in
\mathsurround=2pt

\pagestyle{empty}

\def\ds{\displaystyle}
\def\nd{\noindent}
\def\R{{\mathbb R}}
\def\Z{{\mathbb Z}}
\def\C{{\mathbb C}}
\def\D{{\mathbb D}}
\def\N{{\mathbb N}}
\def\F{{\mathcal F}}
\def\M{{\mathcal M}}

\begin{document}


%\begin{Large}
\begin{large}

\par
\vspace{.15in}

\begin{center}
  {\bf REAL ANALYSIS QUALIFYING EXAM \\
  SPRING 2001}\\
   (Bennett \& Moore) \\
\end{center}


\vspace{.1in}


\begin{large}
\nd {\bf Answer all eight questions.
Throughout, $(X, {\mathcal M} ,\mu)$ denotes a measure space,
$\mu$ denotes a positive measure unless otherwise specified, and all
functions are assumed to be measurable.}
\end{large}

\vspace{.1in}

\renewcommand\baselinestretch{1.25}

\begin{description}

\item[1.]
Let $\nu,\mu$ be positive measures on $(X,\mathcal M)$.
Show that the following are equivalent:

\item[\quad (a)]
$\nu<<\mu$

\item[\quad (b)]
For every $\varepsilon >0$ there exists $\delta>0$ such that
$\nu(B)<\varepsilon$ whenever $B\in \mathcal M$ and
$\mu(B)<\delta$.

\vspace{.05in}

\item[2.]
Suppose $f\in L^p(\R), 1\leq p < \infty$.
Prove that $\ds\lim_{h\to 0}\int_\R |f(x+h)-f(x)|^p\,dx=0$.

\vspace{.05in}

\item[3.]
A set $\mathcal A \subseteq L^1(\mu)$ is called uniformly integrable if
for every  $\varepsilon >0$, there exists a $\delta >0$ such that
$\ds\int_E|f|\,d\mu <\varepsilon$ whenever $\mu(E)<\delta$.

\item[\quad]
Prove Vitali's theorem:

\item[\quad]
If (i) $\mu(X)< \infty$, (ii) $\{f_n\}$ is uniformly integrable, and  
(iii) $f_n\to f$ a.e., where $|f| < \infty$ a.e., then
 $\|f_n-f\|_1 \to 0$.

\vspace{.05in}

\item[4.]
Prove or disprove:  { If $U\subseteq \R$ is open,
then $|\overline U \backslash U|=0$}.

\vspace{.05in}


\item[5.]
Suppose $\{a_n\}$ is a decreasing sequence of positive numbers and
$\ds\sum^\infty_{n=1} a_n<\infty$.
\\
Show that $\ds\lim_{n\to \infty} n a_n=0$.

\vspace{.05in}

\item[6.]
Evaluate $\ds\lim_{A\to\infty} \,\int^A_0 \frac{\sin(x)}{x}\,dx$.
(Hint: $\ds\int^\infty_0 e^{-xt} dt=\frac{1}{x}$).


\vspace{.05in}

\item[7.]
Suppose $T$ is a linear operator on $L^2(X,\mathcal M, \mu)$ with
$\|Tf\|_2=\|f\|_2$.
\\
Show $\langle Tf,Tg \rangle = \langle f,g \rangle$
for all $f,g\in L^2$.


\vspace{.05in}

\item[8.]
Suppose $g_\alpha\in L^2(X, \mathcal M, \mu)$ are such that
$\ds \left|\int_X f(x) g_\alpha(x)\,d\mu(x) \right| \leq \|f\|^3_2$
for all $\alpha$ and $f \in L^2(X, \mathcal M, \mu).$
Show that there exists an $ M>0 $ such that
$\|g_\alpha\|_2 \leq M<\infty $ for every $ \alpha$.

\vfill

\end{description}

%\end{Large}
\end{large}

\end{document}
