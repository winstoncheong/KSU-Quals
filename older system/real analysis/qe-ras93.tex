% latex file
\def\hcorrection#1{\advance\hoffset by #1 }
\def\vcorrection#1{\advance\voffset by #1 }

\documentclass{article}
\usepackage{my,amsxtra,amssymb,amsthm}

\vcorrection{-1.0in}
\hcorrection{-0.8in}
\textwidth 6.0in
\textheight 9.0in

\def\R{\mathbb R}
\def\C{\mathbb C}
\def\N{\mathbb N}
\def\Z{\mathbb Z}
\def\Q{\mathbb Q}

\begin{document}
%\begin{Large}






\begin{center}\begin{LARGE}
{\bf Real Analysis Qualifying Exam}\\ 
{\bf Spring 1993}\\ \end{LARGE}
\end{center}
\vspace{0.1in}
\noindent\hrulefill\\

\begin{description}
\item[1.]
State and prove the monotone convergence theorem for non-negative functions.
Is it true if monotone increasing is replaced by monotone decreasing?
Prove or give a counter-example.

\item[2.]
Discuss the relationship between convergence in measure and convergence
almost everywhere. Prove or give counter-examples for all assertions.

\item[3.]
Suppose $T$ is a linear functional on ${\cal C}(X)$ where $X$ is a compact
Hausdorf space. Show that $T1 = \parallel T \parallel_{op}$ if and only if
$T$ is a positive operator.

\item[4.]
Prove that
$|\int_\Omega fgh d \mu| \leq \left(\int_\Omega |f|^p d\mu \right)^{\frac{1}{p}}
 \left(\int_\Omega |g|^q d \mu \right)^{\frac{1}{q}}
 \left( \int_\Omega |h|^r d\mu \right)^{\frac{1}{r}} \hbox{\ if\ }
 1/p + 1/r + 1/q =1$.

\item[5.]
Suppose $f$ is a real valued nondecreasing function on $[a,b]$ which is
differentiable almost everywhere. Show that $f^\prime$ is Legesgue
measurable. Is it true that $\int^b_a f^\prime(x) dx=f(b) - f(a)$? Prove
or give a counter-example.

\item[6.]
Show that a Banach space where the norm satisfies the parallelogram law,
$\parallel x+y \parallel^2 + \parallel x-y \parallel^2 =
 2\parallel x \parallel^2 + 2\parallel y \parallel^2$, is a Hilbert space.

\item[7.]
Define $f \ast g(x) = \int^\infty_{-\infty} f(x-y) g(y) dy$. Show that
$\parallel f \ast g \parallel_1 \leq \parallel f \parallel_1
 \parallel g \parallel_1$.

\item[8.]
Suppose $(\Omega, {\cal M}, \mu)$ is a measure space and that $\Omega$ is
countable and ${\cal M}$ is the power set of $\Omega$, (i.e. ${\cal M}$ is
the set of all subsets of $\Omega$). Must $L^p \subset L^q$ if
$p \leq q$? Prove or give a counter-example.

\item[9.]
Let ${\cal M}$ be a collection of subsets of a set $\Omega$ with the
following properties:

\item[\quad] i.
$\Omega \in {\cal M}$

\item[\quad] ii.
$A, B \in {\cal M}$ implies that $A-B \in {\cal M}$

\item[\quad] iii.
$A_1, A_2, \dots, A_n, \dots \in {\cal M}$ and
$A_1 \subset A_2 \subset \dots \subset A_n \subset \dots$
implies $\bigcup^\infty_{i=1} A_i \in {\cal M}$.

Show that ${\cal M}$ is a $\sigma$-field (also called a $\sigma$-algebra).





\end{description}    
%\end{Large}
\end{document}














