% latex file
\def\hcorrection#1{\advance\hoffset by #1 }
\def\vcorrection#1{\advance\voffset by #1 }

\documentclass{article}
\usepackage{my,amsxtra,amssymb,amsthm}

\vcorrection{-1.0in}
\hcorrection{-0.8in}
\textwidth 6.0in
\textheight 9.0in

\def\R{\mathbb R}
\def\C{\mathbb C}
\def\N{\mathbb N}
\def\Z{\mathbb Z}
\def\Q{\mathbb Q}
\def\A{\mathbb A}

\begin{document}
%\begin{Large}






\begin{center}\begin{LARGE}
{\bf  Real Analysis Qualifying Exam}\\ 
{\bf Spring 1987}\\ \end{LARGE}
\end{center}
\vspace{0.1in}
\noindent\hrulefill\\

The following numbered items form a list of logically related theorems,
most depending on earlier ones. You are asked to {\it prove} as many of
them as you can. In each case, you may use earlier ones in your proof, even
if you did not suceed in proving the ones you cite.

For $f \in L_1(\R)$ and $\mu \in M(\R)$, define $\widehat f$ and
$\widehat \mu$ on $\R$ by the rules
$$\widehat f(t) = \int^\infty_{-\infty} f(x) e^{-itx} dx$$
and
$$\widehat \mu(x) = \int^\infty_{-\infty} e^{-itx} d \mu (t).$$

\begin{description}
\item[1.]
For such $f$ and $\mu$, we have $\widehat f \in C_0(\R)$ and
$\widehat \mu \in C(\R)$ with
$\parallel \widehat f \parallel_u \leqq \parallel f \parallel_1$
and $\parallel \widehat \mu \parallel_u \leqq \parallel \mu \parallel$.

[HINT: First consider the case that $f$ is the characteristic function of
a bounded interval.]

\item[2.]
If $a \in \R$ and $f(x) = e^{-|x|iax}$ for all $x \in \R$, then
$f \in L_1 (\R)$ and
$$\widehat f (t) = \frac{2}{1+(t-a)^2} \hbox{\ for\ } \hbox{all\ }
  t \in \R.$$

\item[3.]
For $f,g \in L_2 (\R)$, the formula
$$f \ast g(x) = \int^\infty_{-\infty} f(x-u) g(u) du$$
defines a function $f \ast g$ at almost every $x \in \R$,
$f \ast g \in L_1 (\R)$, and
$\parallel f \ast g \parallel_1 \leq \parallel f \parallel_1
 \parallel g \parallel_1$.
[You may use without proof the measurability of the map
$(x,u) \to f(x-u)g(u)$.]

\item[4.]
If $f,g \in L_1 (\R)$, then
$\widehat{f \ast g} (t) = \widehat f(t) \widehat g(t)$ for all
$t \in \R$.

\item[5.]
With pointwise operations, the set $\A(\R) = \{\widehat f:f \in L_1(\R)\}$
is a dense subalgebra of $C_0(\R)$ where $C_0(\R)$ has the uniform norm.

\item[6.]
If $f \in L_1(\R)$, $\mu \in M(\R)$, and $u \in \R$, then
$$\int^\infty_{-\infty} f(x) \widehat \mu (x-u)dx = \int^\infty_{-\infty}
  \widehat f(t) e^{iut} d\mu (t).$$

\item[7.]
If $\mu, \nu \in M(\R)$ and $\widehat \mu (x)= \widehat \nu (x)$ for all
$x \in \R$, then $\mu = \nu$.

For use in the next two problems, define
$M_0(\R) = \{\mu \in M(\R): \widehat \mu \in C_0(\R)\}$.

\item[8.]
If $\nu \in M(\R)$, $\mu \in M_0(\R)$, and $\nu << |\mu|$ ($\nu$ is
absolutely continuous with respect to $\mu$), then $\nu \in M_0(\R)$.

[HINT: Approximate a Radon-Nikodyn derivative in
$L_1(|\mu|)$ by an $\widehat f \in \A (\R)$ and use 6.]

\item[9.]
If $\mu \in M_0 (\R)$, then $\mu(\{a\})=0$ for all $a \in \R$.
[HINT: Otherwise $\delta_a << \mu$.]

\item[10.]
Suppose that $f: \R \to \C$ is bounded and differentiable on $\R$ with
$f^\prime$ also bounded on $\R$. Then for all $g \in L_1(\R)$ the formula
in 3 defines a function $f \ast g$ at each $x \in \R$, $f \ast g$ is
differentiable at every point of $\R$, and
$(f \ast g)^\prime = f^\prime \ast g$.





\end{description}    
%\end{Large}
\end{document}














