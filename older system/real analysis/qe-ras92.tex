% latex file
\def\hcorrection#1{\advance\hoffset by #1 }
\def\vcorrection#1{\advance\voffset by #1 }

\documentclass{article}
\usepackage{my,amsxtra,amssymb,amsthm}

\vcorrection{-1.0in}
\hcorrection{-0.8in}
\textwidth 6.0in
\textheight 9.0in

\def\R{\mathbb R}
\def\C{\mathbb C}
\def\N{\mathbb N}
\def\Z{\mathbb Z}
\def\Q{\mathbb Q}

\begin{document}
%\begin{Large}






\begin{center}\begin{LARGE}
{\bf Real Analysis Qualifying Exam}\\ 
{\bf Spring 1992}\\ \end{LARGE}
\end{center}
\vspace{0.1in}
\noindent\hrulefill\\

In what follows $(X, {\cal A}, \mu)$ is an arbitrary measure space and
$\lambda$ is Lebesgue outer measure on $\R$. Prove all of your assertions.

\begin{description}
\item[1.] (a)
Find a necessary and sufficient condition that $\alpha, \beta \in \C$ satisfy
$|\alpha + \beta| = |\alpha| + |\beta|$.

\item[\quad] (b)
Find a necessary and sufficient condition that $f, g \in L_1 (\mu)$ satisfy.
$$\int |f+g| d\mu = \int |f| d\mu + \int |g| d\mu.$$

\item[2.] (a)
Show why it is impossible to construct a bounded Lebesgue measurable
$f :[0,1] \to \R$ for which there is some $\alpha >0$ such that
$$\int^1_0 |f-g| d \lambda > \alpha$$
for every Riemann integrable $g: [0,1] \to \R$.

\item[\quad] (b)
Can such an $f$ be found if we take $\alpha =0$? Why?

\item[3.] (a)
Construt Lebesgue measurable functions $f_n$, $f:[0,1] \to \C (n=1,2, \dots)$
such that $\lim_{n \to \infty} \int^1_0 |f-f_n| d \lambda = 0$ but we don't
have $f_n \to f \lambda$ - a.e.

\item[\quad] (b)
What positive result about pointwise convergence follows from convergence in
$L_1$-norm?

\item[4.]
Let $\nu$ be a countably additive measure on a (not necessarily $\sigma$)
algebra ${\cal E}$ of subsets of $X$. For $T \subset X$, define
$$\nu^\ast (T) = \inf \left\{\sum^\infty_{n=1} \nu (E_n) : (E_n)^\infty_{n=1}
  \subset {\cal E} \hbox{\ and\ } T \subset \bigcup^\infty_{n=1} E_n
  \right\}.$$
Prove that

\item[\quad] (a)
$\nu^\ast$ is an outer measure on $X$,

\item[\quad] (b)
$E \in {\cal E} \Rightarrow E$ is $\nu^\ast$-measurable,

\item[\quad] (c)
$E \in {\cal E} \Rightarrow \nu^\ast (E) = \nu(E)$.

\item[5.]
Let $F$ be a subset of $[0,1]$ that is not Lebesgue measurable.

\item[\quad] (a)
Is it possible that $\lambda ([0,1] \backslash F) = 0$? Why?

\item[\quad] (b)
Is it possible that $\lambda (F) =1$? Why?

\item[6.]
Let $F$ be any subset of $[0,1]$ (possible nonmeasurable) and let
$\{I_1, I_2, \dots, I_n\}$ be a partition of $[0,1]$ into subintervals. Must
it be true that
$$\lambda (F) = \sum^n_{k=1} \lambda (F \cap I_k)?$$

Why?

\item[7.]
Let $f_n, f: X \to \C$ be ${\cal A}$-measurable with $f_n \to f$ $\mu$-a.e.
as $n \to \infty$. Suppose there exists $g \in L^+_1 (\mu)$ with
$|f_n| \leq g$ $\mu$-a.e. for every $n\geq 1$. Fix $\varepsilon >0$ and
define
$$A_N = \bigcup^\infty_{n=N} \{x \in X: |f(x) - f_n (x)| \geq \varepsilon\}.$$
Prove that $\mu(A_N) \to 0$ as $N \to \infty$.

\item[8.]
Let $C$ be Cantor's ternary set and let $\psi$ be Lebesgue's singular function.
Suppose that $x \in C$ but $x$ is not the left-hand endpoint of an open
interval disjoint from $C$. Prove that
$$D^+ \psi (x) = +\infty.$$
Recall the definiton:
$$D^+ \psi (x) = \overline{\lim_{t\to x^+}}
 \, \frac{\psi (t) - \psi (x)}{t-x}.$$







\end{description}    
%\end{Large}
\end{document}














