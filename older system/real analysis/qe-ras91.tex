% latex file
\def\hcorrection#1{\advance\hoffset by #1 }
\def\vcorrection#1{\advance\voffset by #1 }

\documentclass{article}
\usepackage{my,amsxtra,amssymb,amsthm}

\vcorrection{-1.0in}
\hcorrection{-0.8in}
\textwidth 6.0in
\textheight 9.0in

\def\R{\mathbb R}
\def\C{\mathbb C}
\def\N{\mathbb N}
\def\Z{\mathbb Z}
\def\Q{\mathbb Q}

\begin{document}
%\begin{Large}






\begin{center}\begin{LARGE}
{\bf Real Analysis Qualifying Exam}\\ 
{\bf Spring 1991}\\ \end{LARGE}
\end{center}
\vspace{0.1in}
\noindent\hrulefill\\

In what follows, $(X, {\cal A}, \mu)$ is an arbitrary measure space and
$\lambda$ is Lebesgue measure on $\R$.

\begin{description}
\item[1.](a)
What does it mean to say that a function $f:X \to [-\infty, \infty]$ is
${\cal A}$-measurable?

\item[\quad] (b)
Prove that if ${\cal F}$ is a countable, nonvoid set of such functions $f$
and if
$$g(x) = \sup \{f(x) : f\in {\cal F}\}$$
for each $x \in X$, then $g$ is ${\cal A}$-measurable.

\item[\quad] (c)
Give an example of $X, {\cal A}$, and ${\cal F}$ to show that the assertion in
(b) can fail if ``countable" is omitted.

\item[2.] Let $(E_n)^\infty_{n=1} \subset {\cal A}$ and define
$$E = \{x \in X :\{n \in \N : x \in E_n\} \hbox{\ is\ }
  \hbox{infinite\ }.$$
Prove that $E \in {\cal A}$ and if $\sum^\infty_{n=1} \mu (E_n) < \infty$,
then $\mu (E) = 0$. [Hint: Consider the sets
 $A_j = \cup^\infty_{n=j} E_n$.]

\item[3.]
Let $f \in L_1 (\mu)$ and $\varepsilon >0$. Prove that there is some
$\delta >0$ such that
$$A \in {\cal A}, \mu (A) < \delta \Rightarrow |\int_A fd\mu| <
 \varepsilon.$$

\item[4.]
Suppose $f \in L_p (\R)$ and $p>0$. Prove that
$$\lim_{t \to 0} \int_\R |f(x+t) - f(x) |^p dx =0.$$
[Hint: Fist approximate $f$ with a continuous function having compact
support.]

\item[5.]
Suppose $\mu(X) < \infty, f: X \to [0, \infty]$ is ${\cal A}$-measurable,
$\int_X fd\mu < \infty$, and ${\cal B}$ is a sub-$\sigma$-algebra of
${\cal A}$. Prove that there exists a ${\cal B}$-measurable
$g:X \to [0,\infty]$ such that
$$\int_B gd\mu = \int_B fd\mu \quad \forall B \in {\cal B}.$$
Prove also that
$$\int_X hgd\mu = \int_X hfd\mu$$
whenever $h:X \to [0,\infty]$ is ${\cal B}$-measurable.

\item[6.]
Suppose $(g_n)^\infty_{n=1} \subset L_1 ([0,1])$, $g_n \geq 0$ a.e.
$\forall n$, and the sequence $(\int^1_0 fg_nd \lambda)^\infty_{n=1}$
converges $\forall f \in C([0,1])$. Prove that there is a Borel measure
$\nu$ on [0,1] such that
$$\lim_{n \to \infty} \int^1_0 fg_n d\lambda = \int_{[0,1]} fd\nu
  \quad \forall f \in C ([0,1]).$$

\item[7.]
For $f \in L_1 (\R)$, define its Fourier transform $\widehat f$ on $\R$ by
$\widehat f (t)= \int_\R f(x)e^{-itx} dx$. Prove that if $f,g \in L_1 (\R)$,
then

\item[\quad] (a)
$\widehat f$ is continuous on $\R$,

\item[\quad] (b)
$\widehat f$ is bounded,

\item[\quad] (c)
$\lim_{|t| \to \infty} \widehat f (t) = 0$
[Hint: First suppose $f$ is a step function.],

\item[\quad] (d)
$\int_\R f(x) \widehat g (x) dx = \int_\R \widehat f (t) g(t) dt.$

\item[8.]
Suppose that $g: \R \to \C$ is measurable and
$$\int_\R (1+|y|) |g(y)| dy < \infty.$$

Define $f$ on $\R$ by
$$f(x) = \int_\R g(y) \cos (xy) dy.$$

Prove that $f$ is differentiable on $\R$ and
$$f^\prime (x) = -\int_\R yg(y) \sin (xy) dy$$
for every $x \in \R$.

\item[9.]
Give an explicit example of a Borel measure $\sigma$ on $\R$ for which

\item[\quad] (a)
$\sigma (\R) =1$,

\item[\quad] (b)
$\sigma (\{x\}) = 0 \forall x \in \R$, and

\item[\quad] (c)
for some compact set $P \subset \R$ we have $\lambda (P) = 0$ and
$\sigma (P) =1$ where $\lambda$ is Lebesgue measure on $\R$.

\item[10.]
Suppose that $f:[a,b] \to \C$ is absolutely continuous. Prove that the total
variation $V^b_a f$ over $[a,b]$ of $f$ is given by
$$V^b_a f= \int^b_a |f^\prime (x)| dx.$$
[Hint: Approximate with appropriate step functions.]






\end{description}    
%\end{Large}
\end{document}














