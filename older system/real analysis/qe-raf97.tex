% latex file
\def\hcorrection#1{\advance\hoffset by #1 }
\def\vcorrection#1{\advance\voffset by #1 }

\documentclass{article}
\usepackage{my,amsxtra,amssymb,amsthm}

\vcorrection{-1.0in}
\hcorrection{-0.8in}
\textwidth 6.0in
\textheight 9.0in
\begin{document}
%\begin{Large}






\begin{center}\begin{LARGE}
{\bf Real Analysis  Qualifying Exam}\\ 
{\bf Fall 1997}\\ \end{LARGE}
\end{center}
\vspace{0.1in}
\noindent\hrulefill\\
Unless otherwise stated, $(X, {\cal A}, \mu)$ is an arbitrary measure
space.

\begin{description}
\item[1.]
Let $E \subset \Bbb R$ be a Borel set. Prove that
$$E^\prime = \{(x,y) \in \Bbb R^2 : x + y \in E\}$$
is a Borel set.

\item[2.]
Let $f$ be a measurable function on $X$ and $p>0$. Prove
$$\int |f|^p d\mu = p \int^\infty_0 t^{p-1} \mu (\{ |f| > t \})dt.$$

\item[3.]
Suppose that $f_n$, $f \in L^1(\mu)$ and
$\parallel f - f_n \parallel_1 \to 0$. Prove
$$\lim \sup \int \log |f_n| d \mu \leq \int \lim \sup \log |f_n| d \mu,$$
where $\log x = -\infty$ for $x=0$.

\item[4.]
Suppose that $\mu$ is a positive finite Borel measure on a Hausdorff space
$X$, and for each open set $V \subset X$,
$$\mu (V) = \sup \{\mu (K) : K \hbox{\ is\ } \hbox{\ compact\ },
K \subset V \}.$$

Prove: Given a Borel set $E \subset X$ and $\varepsilon >0$, there exist a
compact set $K$ and an open set $V$ such that
$$K \subset E \subset V \hbox{\ and\ } \mu(V \backslash K) < \varepsilon.$$

\item[5.]
If $f$ is complex function on $[a,b]$, its total variation
$F : [a,b] \to [0, \infty]$ is defined by
$$F(x) = \sup \left\{ \sum^N_{j=1} \left| f(t_j) - f(t_{j-1}) \right|:
a = t_0 < t_1 < \dots < t_N= x \right\}.$$

Given an example of a continuous function $f$ on $[a,b]$ for which
$F(x)= \infty$ for any $x \in (a,b]$.

\item[6.]
Let $f_n(t) = e^{i(n+ \frac{1}{2})t}, t \in [0,2\pi]$. Is it true that
$\{f_n\}_{n \in \Bbb Z}$ is a complete orthonormal system in
$L^2[0,2\pi]$?

(You may use the fact that the functions $e_n$, $e_n(t) = e^{int}$,
$n \in \Bbb Z$, form a complete orthonormal system in $L^2[0,2\pi]$.)

\item[7.]
Construct a nonzero positive Borel measure $\mu$ on $[0,1]$ which is singular
with respect to Lebesgue measure and such that $\mu (\{t\}) = 0$
for any $t \in [0,1]$.

\item[8.]
Let $\{f_n\}$ be a sequence of functions in $L^1[0,1]$ such that
$f_n(t) \to 0$ for any $t \in [0,1]$. Is it true that
$\int_0^1 f_n(t) dt \to 0$?





\end{description}    
%\end{Large}
\end{document}














