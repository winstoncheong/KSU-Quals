% latex file
\def\hcorrection#1{\advance\hoffset by #1 }
\def\vcorrection#1{\advance\voffset by #1 }

\documentclass{article}
\usepackage{my,amsxtra,amssymb,amsthm}

\vcorrection{-1.0in}
\hcorrection{-0.8in}
\textwidth 6.0in
\textheight 9.0in

\def\R{\mathbb R}
\def\C{\mathbb C}
\def\N{\mathbb N}
\def\Z{\mathbb Z}
\def\Q{\mathbb Q}

\begin{document}
%\begin{Large}






\begin{center}\begin{LARGE}
{\bf Real Analysis Qualifying Exam}\\ 
{\bf Spring 1994}\\ \end{LARGE}
\end{center}
\vspace{0.1in}
\noindent\hrulefill\\

Throughout $(X, {\cal M}, \mu_)$ denotes a measure space, $\mu$ denotes a
positive measure unless otherwise specified, and all functions are
assumed to be measurable.

\begin{description}
\item[1.]
Suppose $\mu(X) < \infty$ and that $f_n \to f$ in measure. Prove that there
exists a subsequence $\{ f_{n_k} \}$ of $\{f_n\}$ such that $f_{n_k} \to f$
a.e.

\item[2.]
Let $g$ be a bounded function which has the property that for every measurable
set $E$, $\lim_{n\to \infty} \int_E g(nx) dx =0$. Show that for every
$f \in L^1 (X, {\cal M}, \mu), \lim_{n \to \infty} \int_X f(x) g(nx) dx =0$.

\item[3.]
Suppose $f \in L^1 (\mu)$, $g \in L^p (\mu)$, $1 \leq p \leq \infty$. Prove
that
$\parallel f \ast g \parallel_p \leq \parallel f \parallel_1 \parallel g
  \parallel_p$.

\item[4.]
Show that any orthonormal set in a separable Hilbert space is at most
countable.

\item[5.] (a)
Construct a closed set $K \subseteq [0,1]$ such that $|K| > \frac{1}{2}$ and
$K$ contains no rational.

\item[\quad] (b)
Can you construct such a $K$ so that $|K \cap I| \leq \frac{9}{10} |I|$
for every interval $I \subseteq [0,1]$?

\item[6.]
Suppose $T:B \to C$ is a bounded linear tranformation between the Banach
spaces $B$ and $C$. Note that $T$ induces a map $T^\ast : C^\ast \to B^\ast$
given by $T^\ast (f) = f \circ T$. Hence, this induces a map
$T^{\ast \ast} : B^{\ast \ast} \to C^{\ast \ast}$. Consider also the natural
embedding $i_B : B \hookrightarrow B^{\ast \ast}$ given by $b \to \widehat b$
where $\widehat b (S) = S(b)$ for $S \in B^\ast$. Prove that
``$T^{\ast \ast} |_B = T$", that is, show that
$T^{\ast \ast} \circ i_B = i_c \circ T$ where $i_c$ is the natural embedding
$i_c :C \hookrightarrow C^{\ast \ast}$.

\item[7.]
Suppose $\{f_n\}$ is a sequence of functions which has $f_n \to f$ a.e. and
$\parallel f_n \parallel_1 \to \parallel f \parallel_1 < \infty$. Prove
that $f_n \to f$ in $L^1$.

\item[8.]
Let $D$ be the unit disk in the complex plane. Assume the following facts
from complex analysis:

\item[\quad] 1.
Given any continuous function $f$ on $\partial D$, there exists a unique
harmonic function $u$ on $D$, continuous on $\overline D$, such that
$u|_{\partial D} = f$.

\item[\quad] 2.
The sum of any two harmonic functions is harmonic, as is the multiplication
of harmonic function by a constant.

\item[\quad] 3.
If $u$ is assoiciated to $f$ as in 1, then
$$\sup_{z \in D} |u(z)| \leq \sup_{z \in \partial D} |f(z)|.$$

\item[\quad] 4.
If $f$ is real valued, so is $u$. If $f \equiv 1$ then $u \equiv 1$.

Fix $z_0 \in D$. Prove that there exists a positive measure $w_{z_0}$ on
$\partial D$ such that
$$u(z_0) = \int_{\partial D} f(z) dw_{z_0} (z)$$
if $f$ and $u$ are related as in 1.

Remark: $w_{z_0}$ is called harmonic measure. You don't even have to know the
defintion of harmonic function to solve this problem.

\item[9.]
Let $\{a_n\} \in \ell^2$. Prove that
$$\exp \left(\frac{1}{2 \pi} \int^\pi_{-\pi} \log \left| \sum^\infty_{n=1}
  a_n e^{in\theta} \right| d \theta \right) \leq
  \left(\sum^\infty_{n=1} |a_n|^2 \right)^{\frac{1}{2}}$$

\item[10.]
Suppose $f_n(x)$ is a sequence of functions on the interval [0,1] which
satisfy the distrubution estimate:
$$|\{x \in [0,1] \parallel f_n (x) | > \lambda \} | \leq e^{-\lambda / n}.$$
Prove that $\lim_{n \to \infty} \sup \frac{|f_n (x)|}{n \log n} \leq 1$
a.e.






\end{description}    
%\end{Large}
\end{document}














