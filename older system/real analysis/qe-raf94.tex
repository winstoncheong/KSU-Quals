% latex file
\def\hcorrection#1{\advance\hoffset by #1 }
\def\vcorrection#1{\advance\voffset by #1 }

\documentclass{article}
\usepackage{my,amsxtra,amssymb,amsthm}

\vcorrection{-1.0in}
\hcorrection{-0.8in}
\textwidth 6.0in
\textheight 9.0in
\begin{document}
%\begin{Large}






\begin{center}\begin{LARGE}
{\bf Real Analysis Qualifying Exam}\\ 
{\bf Fall 1994}\\ \end{LARGE}
\end{center}
\vspace{0.1in}
\noindent\hrulefill\\

\begin{description}
\item[1.]
Define Lebesgue outer measure, $\lambda^\ast$, on [0,1].

\item[2.]
Let $(X,S, \mu)$ be a finite measure space and suppose that
$\{f_n\}$ is a sequence of measurable functions which converges to a finite
function $f$ a.e.. Let $\varepsilon > 0$ and set
$A_N = \{x : \sup_{n \geq N} |f_n(x) - f(x) | > \varepsilon \}$.
Prove that $\mu(A_N) \to 0$ as $N \to \infty$.

\item[3.]
Suppose $f \in L^p ([0,\infty))$ where $1 \leq p \leq 2$. For $x \geq 0$
set $g(x) = \int^{x^2}_x f(t) dt$. Show that
$\lim_{x \to \infty} \frac{g(x)}{x} = 0$.

\item[4.]
Let $f$ be a $C^\infty$ function (that is, $f$ and all its derivatives
exist and are continuous) from $\Bbb R$ to $\Bbb R$ with the property that
for each $x \in \Bbb R$ there exists a $k \in \Bbb N$ (depending on $x$)
such that $\frac{\partial^k f}{\partial x^k} (x) =0$. Show that there
exists an interval $I \subseteq \Bbb R$, $I \neq \varnothing$
such that $f|_I$ is a polynomial.

\item[5.] (a)
Construct a bounded, Lebesgue integrable function $g(x)$ on [0,1] such that
$\int^1_0 |f(x) - g(x)| dx > 0$ for every Riemann integrable function
$f(x)$ on [0,1].

\item[\quad] (b)
Can you construct such a $g(x)$ so that $\int^1_0 |f(x) - g(x)| dx > 10^{-5}$
for every Riemann integrable function $f(x)$ on [0,1]?

\item[6.]
Suppose $f$ is a Lebesgue measurable function on [0,1]. Is it true that if
$f^\prime = 0$ a.e. then $f$ cannot be strictly increasing?

\item[7.]
Suppose $\Phi : [0, \infty) \to [0, \infty)$ is strictly increasing
continuously differentiable function with $\Phi(0) = 0$. Suppose that
$f \in L^1 (X, {\mathcal M}, \mu)$ and for $\lambda > 0$ set
$m(\lambda) = \mu(\{x \in X : |f(x)| > \lambda \})$. Prove that
$\int_X \Phi (f(x)) d \mu (x) = \int^\infty_0 m(\lambda) d \Phi (\lambda)$.

\item[8.]
Suppose $\mu$ is a complex measure on $\Bbb R^n$ with the property that
$\int_{\Bbb R^n} f(x) d\mu \geq 0$ whenever $f \geq 0$ is a continuous
function on $\Bbb R^n$ with compact support. Show that $\mu$ is a positive
measure.

\item[9.]
Let $v, \mu$ be complex measures on $(X,{\mathcal M})$. Suppose $v << \mu$. Show that
for every $\varepsilon >0$ there exists a $\delta > 0$ such that if
$|\mu|(E) < \delta$ then $|v(E)| < \varepsilon$.

\item[10.]
Suppose $(X,{\mathcal M}, \mu)$ is a measure space and $\mu$ is positive and
$\sigma$-finite.

\item[\quad] (a)
Suppose $f \in L^1(X, {\mathcal M}, \mu)$ and suppose
${\mathcal A} \subseteq {\mathcal M}$ is also a
$\sigma$-algebra. Prove that there exists a function
$g \in L^1(X,{\mathcal A}, \mu)$
such that $\int_E gd\mu = \int_E fd \mu$ for all $E \in {\mathcal A}$.

\item[\quad] (b)
If $g$ and $f$ are related as in part (a) we write $g=E(f|{\mathcal A})$.
Suppose that ${\mathcal B} \subset {\mathcal A}$ is also a $\sigma$-algebra
on $X$. Show that $E(E(h|{\mathcal A}) |{\mathcal B}) = E(h|{\mathcal B})$
whenever $h \in L^1(X, {\mathcal M}, \mu)$.

\end{description}    
%\end{Large}
\end{document}














