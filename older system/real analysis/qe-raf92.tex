% latex file
\def\hcorrection#1{\advance\hoffset by #1 }
\def\vcorrection#1{\advance\voffset by #1 }

\documentclass{article}
\usepackage{my,amsxtra,amssymb,amsthm}

\vcorrection{-1.0in}
\hcorrection{-0.8in}
\textwidth 6.0in
\textheight 9.0in

\def\R{\mathbb R}
\def\C{\mathbb C}
\def\N{\mathbb N}
\def\Z{\mathbb Z}
\def\Q{\mathbb Q}

\begin{document}
%\begin{Large}






\begin{center}\begin{LARGE}
{\bf Real Analysis Qualifying Exam}\\ 
{\bf Fall 1992}\\ \end{LARGE}
\end{center}
\vspace{0.1in}
\noindent\hrulefill\\

In what follows $(X, {\cal A}, \mu)$ is an arbitrary measure space and
$\lambda$ is Lebesgue outer measure on $\R$.

\begin{description}
\item[1.](a)
Find a necessary and sufficient condition that $\alpha, \beta \in \C$
satisfy $|\alpha + \beta| = |\alpha| + |\beta|$.

\item[\quad] (b)
Find a necessary and sufficient condition that $f, g \in L_1 (\mu)$
satisfy:
$$\int |f+g| d \mu = \int |f| d \mu + \int |g| d \mu.$$

\item[2.]
Suppose
$$\sum^\infty_{n=1} \mu (A_n) < \infty.$$
Show that
$$A=\{x:x \in A_n \hbox{\ for\ } \hbox{infinitely\ } \hbox{many\ } n \}$$
has $\mu(A) =0$.

\item[3.]
Let $F$ be a subset of $[0,1]$ that is not Lebesgue measureable.

\item[\quad] (a)
Is it possible that $\lambda (F) =0$? Why?

\item[\quad] (b)
It is possible that $\lambda (F) =1$? Why?

\item[4.]
Suppose $f$ is continuous on [0,1]. Show that
$$\hbox{Riemann\ } \int^1_0 fdx = \hbox{\ Lebesgue\ } \int^1_0 fdx.$$

\item[5.]
Let $\nu$ be a finite measure that is absolutely continuous with respect to a
measure $\mu$ that is regular. Prove that $\nu$ is regular.

\item[6.] (a)
Fix $\alpha >0$. Show that it is impossible to construct a bounded, Lebesgue
measureable function $f$ such that
$$\int^1_0 |f-g| d \lambda > \alpha$$
for every Riemann integrable function $g$.

\item[\quad] (b)
Can such a $f$ be found if we take $\alpha =0$? Why?

\item[7.]
Let $X,Y$ be topological spaces, each having a countable base for its
topology. Let
$${\cal B} (X) \times {\cal B}(Y)$$
be the smallest $\sigma$-algebra of subsets of $X \times Y$ that contains
every $R \times S$ where $R \in {\cal B} (X)$ and $S \in {\cal B}(Y)$.
Prove that
$${\cal B}(X) \times {\cal B}(Y) = {\cal B}(X \times Y).$$

\item[8.]
State which types of convergence imply which other types of convergence. Prove
all of your assertions and prove counterexamples.

\item[\quad] (A)
Pointwise a.e. convergence

\item[\quad] (B)
Convergence in $L^1$

\item[\quad] (C)
Convergence in measure






\end{description}    
%\end{Large}
\end{document}














