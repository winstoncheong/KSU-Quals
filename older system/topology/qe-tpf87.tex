% latex file
\def\hcorrection#1{\advance\hoffset by #1 }
\def\vcorrection#1{\advance\voffset by #1 }

\documentclass{article}
\usepackage{my,amsxtra,amssymb,amsthm}

\vcorrection{-1.0in}
\hcorrection{-0.8in}
\textwidth 6.0in
\textheight 9.0in

\def\R{\mathbb R}
\def\C{\mathbb C}
\def\N{\mathbb N}
\def\Z{\mathbb Z}
\def\Q{\mathbb Q}

\begin{document}
%\begin{Large}






\begin{center}\begin{LARGE}
{\bf Topology Qualifying Exam}\\ 
{\bf Fall 1987}\\ \end{LARGE}
\end{center}
\vspace{0.1in}
\noindent\hrulefill\\

Do 9 of the following 15 problems.

In the following problems, let $\R$ denote the real line with the usual
topology, and let $\N$ denote the natural numbers.

\begin{description}
\item[1.]
Prove that any continuous bijection $f : \R \to \R$ is a homeomorphism.

\item[2.]
Prove that in a metrizable space $X$ without isolated points, the closure of
a discrete set in $X$ must be nowhere dense in $X$.

\item[3.] (a)
Prove that every closed subset of a metrizable space $X$ is a $G_\delta$ in
$X$.

\item[\quad] (b)
Give an example to show that a closed subset of a Hausdorff space $X$ is not
necessarily a $G_\delta$ in $X$.

\item[4.] (a)
Characterize the compact subsets of $\R$ and prove that your characterization
is correct.

\item[\quad] (b)
State and prove a maximum value theorem from calculus.

\item[\quad] (c)
Using maximum, minimum, and intermediate value theorems from calculus, prove
that every continuous open function $f : [0,1] \to [0,1]$ is surjective.

\item[5.]
Let $X$ be the topological space whose underlying set is the set of real
numbers and whose topology has as a basis the set of half open intervals
of the form $[a,b)$ where $a < b$. Show that $X$ is

\item[\quad] (a)
first countable.

\item[\quad] (b)
separable.

\item[\quad] (c)
not second countable.

\item[\quad] (d)
not metrizable.

\item[6.] (a)
True-False

\item[\qquad] (i)
The composition of quotient maps is a quotient map.

\item[\qquad] (ii)
The product of metrizable spaces is metrizable.

\item[\qquad] (iii)
$f: X \to Y$ is a topological embedding iff $f$ is one-to-one and $X$ has
the coarsest (=weakest) topology making $f$ continuous.

\item[\qquad] (iv)
A space is $T_1$ iff it is locally $T_1$; i.e., each point has a base of
$T_1$ neighborhoods.

\item[\qquad] (v)
A space is $T_2$ iff it is locally $T_2$; i.e., each point has a base of
$T_2$ neighborhoods.

\item[\qquad] (vi)
Every metrizable space is normal.

\item[\qquad] (vii)
Every locally compact Hausdorff space is completely regular.

\item[\qquad] (viii)
Every subspace of a separable Hausdorff space is separable and Hausdorff.

\item[\quad] (b)
For each false entry, give a counter example (no proofs).

\item[7.]
Let $S^1$ denote the unit circle in the plane (with the usual topology). Give
4 examples, 1 compact, 1 non-compact, 1 non-locally connected, 1 non-locally
compact, of spaces homotopically equivalent to $S^1$, but not homeomorphic
to $S^1$.

\item[8.]
Let $f: A \to B$ where $A, B \subseteq \R$. For $x \in \R$, define
$$\hbox{osc} (f, x) = \inf \{\hbox{diam\ } f (A \cap U) | \ x \in U \quad
  \hbox{open\ } \hbox{in\ } \R\}$$
and
$$A^\ast = \{x \in  \overline A | \hbox{osc} (f,x) = 0\}.$$
Prove that if $f: A \to B$ is an order preserving homeomorphism and
$A, B$ are dense in $\R$, then $A^\ast = \R$.

\item[9.]
Prove that if $f : [0,1] \to X$ is a continuous open surjection onto a
nondegenerate Hausdorff space $X$, then $X$ is homeomorphic to $[0,1]$.

\item[10.]
Prove that if a filter $\mathcal F$ is contained in a unique ultrafilter
$\mathcal G$, then $\mathcal F = \mathcal G$.

\item[11.]
Prove that the following two statements about a $T_1$-space $X$ are
equivalent.

\item[\quad] (a)
Every infinite subset of $X$ has an accumulation point in $X$.

\item[\quad] (b)
At least one member of every infinite open cover of $X$ can be discarded with
the remaining sets still covering $X$.

\item[12.]
Find the specific error in the following: ``Proof" that  the uncountably
infinite power of a two point discrete space is metrizable. Let $D=\{0,1\}$
have the discrete topology. For each $r \in \R$ define
$$f_r: D^N \to D \hbox{\ by\ } f_r(g) =
        \begin{cases}
          g(r) \quad &\hbox{if\ } r\in \N \\
          0 \quad &\hbox{if\ } r \in \R - \N.
          \end{cases}$$
Then these functions are continuous and thus induce a continuous embedding
$F: D^N \to D^R$. Let $U$ be open in $D^R$. Then $U$ restricts only finitely
may coordinates. Thus $U \cap F[D^N] \neq \emptyset$, so $F[D^N]$ is
dense in $D^R$. But $D^N$ is homeomorphic with the Cantor space, so
$F[D^N]$ is compact. Since $D^R$ is Hausdorff, $F[D^N]$ must be closed in
$D^R$. Hence $F[D^N] = D^R$, and since $D^N$ is metrizalbe, $D^R$ must
be metrizable.

\item[13.]
Prove that if $X$ is compact Hausdorff, then each quasicomponent of $X$
is connected.

\item[14.]
Prove that if ${\mathcal C} = \{C_\alpha | \ \alpha \in \Lambda\}$ is a
family of compact subsets of a Hausdorff space such that the finite intersections
of members of $\mathcal C$ are connected, then $\bigcap {\mathcal C}$ is
connected.

\item[15.]
Show that every connected, locally compact, paracompact Hausdorff space is
Lindel\"of.





\end{description}    
%\end{Large}
\end{document}














