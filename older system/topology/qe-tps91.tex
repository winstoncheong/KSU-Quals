% latex file
\def\hcorrection#1{\advance\hoffset by #1 }
\def\vcorrection#1{\advance\voffset by #1 }

\documentclass{article}
\usepackage{my,amsxtra,amssymb,amsthm}

\vcorrection{-1.0in}
\hcorrection{-0.8in}
\textwidth 6.0in
\textheight 9.0in

\def\R{\mathbb R}
\def\C{\mathbb C}
\def\N{\mathbb N}
\def\Z{\mathbb Z}
\def\Q{\mathbb Q}
\def\P{\mathbb P}
\begin{document}
%\begin{Large}






\begin{center}\begin{LARGE}
{\bf Topology Qualifying Exam}\\ 
{\bf Spring 1991}\\ \end{LARGE}
\end{center}
\vspace{0.1in}
\noindent\hrulefill\\

Work 6 of the following problems. Start each problem on a new sheet of
paper. Do not turn in more than 6 problems.

\begin{description}
\item[1.]
Prove that a continuous map from a compact space to a Hausdorff space
is closed.

\item[2.] (a)
True - False

\item[\qquad] 1.
Every connected space is locally connected.

\item[\qquad] 2.
For $i \in \{ 0,1,2,3,4\}$ every product of $T_i$-spaces is $T_i$.

\item[\qquad] 3.
For $i \in \{0,1,2,3,4\}$ every closed subspace of a $T_i$-space
is $T_i$.

\item[\qquad] 4.
If $(X_i \stackrel{f_i}{\longrightarrow} Y_i)_{i \in I}$
is a set of topological
embeddings, then $\Pi_I X_i \stackrel{\Pi f_i}{\longrightarrow} \Pi_IY_i$
is an embedding.

\item[\qquad] 5.
The statement ``The product of any family of nonempty sets is nonempty"
is equivalent to the Axiom of Choice.

\item[\quad] (b)
For each false entry give a counterexample (no proofs).

\item[3.]
Show that no two of the intervals of $\R [0,1], (0,1)$, and $[0,1)$
(with their usual subspace topologies) are homeomorphic.

\item[4.]
Prove that if $x$ is any point of a compact Hausdorff space, then $x$
has a neighborhood base consisting of closed sets.

\item[5.]
Let $f : X\to Y$ be a quotient map, and assume that $X$ is locally pathwise
connected (i.e., each point has a neighborhood base consisting of pathwise
connected sets). Prove that $Y$ is locally pathwise connected.

\item[6.] (a)
Consider a function $f: X \to \Pi_{\alpha \in A} Y_\alpha$ and the family
of associated coordinate functions $f_\alpha : X \to Y_\alpha$.
Prove that $f$ is continuous if and only if every $f_\alpha$ is continuous,
assuming we give the product set the product topology.

\item[\quad] (b)
Give a counterexample to the above statement if we give the product set the
``box" topology (where all products of open sets are open).

\item[7.]
Prove or disprove: If $f : X\to Y$ is one-to-one and continuous and
$A \subseteq X$, then
$f[Fr (A)] \subseteq Fr (f[A])$, where
$Fr(A) = \overline A \cap \overline{X-A}$.

\item[8.]
Let $X$ be the subspace of the plane $(\R \times \R)$ that consists of all
lines parallel to the $x$-axis that cross the $y$-axis at positive integral
heights, i.e.,
$$X=\{a,b) | a \in \R, b \in \Z, \hbox{\ and\ } b \geq 1\},$$
and let $Y$ be the subspace of the plane that consists of all lines
through the origin that have positive integral slopes, i.e.,
$$Y = \{(0,0)\} \cup \{(a,b) \in \R \times \R | \frac{b}{a} \in \Z
  \hbox{\ and\ } \frac{b}{a} \geq 1\}.$$
Find an error in the following ``proof" that $Y$ is a quotient space of $X$:

Define $f:X \to Y$ by $f(a,b) = (a, ab)$. [Note that the restriction of $f$
to the horizontal line at height $n$ maps this line homeomorphically onto
the line contained in $Y$ of slope $n$]. $f$ is clearly continuous on each
of the lines that make up $X$, so that since $X$ is the disjoint union of
these lines, $f$ is continuous on $X$. Also, since the restriction of $f$ to
each of the lines that make up $X$ is a homeomorphism (and so is open) $f$
is open. Since every open continuous surjection is a quotient map, $f$ must
be a quotient map.

\item[9.]
Prove that a paracompact Hausdorff space is normal.

\item[10.]
Let $\P$ be the irrational numbers with the usual (subspace) topology.
Show that the intersection of any countable family of dense open subsets
of $\P$ is dense in $\P$.

\item[11.]
Prove that the product of connected spaces is connected.

\item[12.]
Let $S$ be a well-ordered uncountable set that has the property that for
every $x \in S$ the subset
$S_x = \{y \in S | y< x\}$ is countable. Give $S$ the order topology. Prove
two of the following:

\item[\quad] (a)
$S$ is first countable.

\item[\quad] (b)
$S$ is {\bf not} second countable.

\item[\quad] (c)
$S$ has the property that each of its infinite subsets has a limit point.

\item[\quad] (d)
$S$ is normal.



\end{description}    
%\end{Large}
\end{document}














