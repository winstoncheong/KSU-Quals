\documentclass[12pt]{article}
\usepackage{amsxtra,amssymb,amsthm,amsmath,latexsym}

\pagestyle{empty}

\oddsidemargin -.5in\evensidemargin -.5in
\topmargin -0.75in \textwidth 7.0in\textheight 9in

\def\R{{\mathbb R}}
\def\ds{\displaystyle}

\begin{document}

\begin{large}

\begin{center}
{\bf  Topology Qualifying Exam}\\
{\bf Spring 2003} -- Miller \& Maginnis \\
\end{center}

\hfill {\bf NAME:}\rule{2.5in}{.01in}



\vspace{-.2in}
\begin{description}

%\vspace{.1in}
\item[1.] Let $X$ be a topological space which is connected and locally path
connected. Prove that $X$ is path connected.


\vspace{.2in}\item[2.] Let $\{X_a|a\in I\}$ be a collection of connected
topological spaces. Prove that $\prod_{a\in I}X_a$ with the product topology 
is connected.


\vspace{.2in}\item[3.] Prove that a compact Hausdorff space is metrizable if
and only if it is second countable.


\vspace{.2in}\item[4.] Let $X$ be a well ordered set with the order topology.
Assume that $X$ has a maximal element. Prove that $X$ is compact.


\vspace{.2in}\item[5.] Prove that a metric space is sequentially compact if
and only if it is limit point compact.


\vspace{.2in}\item[6.] a) Let $X$ be a topological space and $x_0$ an element
of $X$. Define, in detail, what we mean by $\pi_1(X,x_0)$, the fundamental
group of $X$ relative to $x_0$.


\vspace{.2in}\item[\quad]  b) Prove that the construction you gave in part a)
defines a covariant functor from the category of pointed topological spaces
to the category of groups.


\vspace{.2in}\item[\quad]  c) If $X$ is path connected, $x_0\in X$ and
$x_1\in X$ show that $\pi_1(X,x_0)$ is isomorphic to $\pi_1(X,x_1)$.


\vspace{.2in}\item[\quad]  d) Suppose that $X$ and $Y$ are path connected,
$x_0\in X, y_0\in Y$ and $\pi_1(X,x_0)$ is not isomorphic to $\pi_1(Y,y_0)$.
Show that $X$ and $Y$ cannot be homeomorphic.




\end{description}
\end{large}
\end{document}





