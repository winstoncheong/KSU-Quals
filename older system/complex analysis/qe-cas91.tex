% latex file
\def\hcorrection#1{\advance\hoffset by #1 }
\def\vcorrection#1{\advance\voffset by #1 }

\documentclass{article}
\usepackage{my,amsxtra,amssymb,amsthm}

\vcorrection{-1.0in}
\hcorrection{-0.8in}
\textwidth 6.0in
\textheight 9.0in
\begin{document}
%\begin{Large}






\begin{center}\begin{LARGE}
{\bf Complex Variables Qualifying Exam}\\ 
{\bf Spring 1991}\\ \end{LARGE}
\end{center}
\vspace{0.1in}
\noindent\hrulefill\\
In what follows $\Bbb R$ is the real numbers, $\Bbb C$ the complex numbers,
$\Bbb D$ the open unit disc centered at 0 and $\Bbb T$ the boundary of
$\Bbb D$. The set of functions holomorphic in $\Omega$ is denoted
$H (\Omega)$.

\begin{description}
\item[1.]
Let $f$ be a continuous function on $\Bbb D$ which satisfies
$\int_\Delta f=0$ for each triangle $\Delta \subset \Bbb D$ such that one
side on $\Delta$ lies on $\Bbb R$ and another side of $\Delta$ is parallel
to $i\Bbb R$. Show that $f$ is holomorphic.

\item[2.]
Prove that $1 / z$ is not uniformly approximable on $\Bbb T$ by polynomials
in $z$.

\item[3.]
Let $\Omega$ be a region in $\Bbb C, f \in H(\Omega) \backslash \{0\}$, $n$
a positive integer. Suppose that $|z|^n f(z)|$ attains a maximum over
$\Omega$ at some point of $\Omega$. Show that $0 \notin \Omega$.

\item[4.]
Prove that for all $z$ in the open right half-plane $\Bbb H$ the integral
$\int^\infty_1 e^{-t} t^{z-1} dt$ exists and defines a holomorphic
function of $z \in \Bbb H$.

\item[5.] (i)
If $f$ is holomorphic in a neighborhood of $\overline{\Bbb D}$, then
$$|f(z)| \leq \frac{1}{\sqrt{1-|z|^2}} \left[\frac{1}{2\pi}
  \int^\pi_{-\pi} \left| f(e^{i\theta}) \right|^2 d \theta \right]^{1/2}
  \quad \forall z \in \Bbb D.$$

\item[\quad] (ii)
Use (i) to draw the same conclusion in case $f$ is only continuous on
$\overline{\Bbb D}$, holomorphic in $\Bbb D$.

\item[6.]
$f$ and $g$ are each holomorphic in a neighborhood of 0, $f(0) =0$ with
multiplicity $m, g(0) = 0$ with multiplicity $n$. What is the multiplicity of
0 as a zero of $f \circ g$?

\item[7.] (i)
Use the function $f(t) := e^{it}, t \in [0, 2\pi]$, to show that the
Mean-Value Theorem of differential calculus fails (generally) for
complex-valued functions.

\item[\quad] (ii)
Prove that, in spite of (i), if $F$ is holomorphic in a convex region
$\Omega$ and $|F^\prime | \leq M$, then
$$|F(z_2) - F(z_1)| \leq M |z_2 - z_1| \quad \forall z_1, z_2
  \in \Omega.$$

\item[8.]
Let $\Omega$ be a bounded region in $\Bbb C$,
$f :\overline \Omega \to \Bbb C$ a continuous non-constant function
which is holomorphic in $\Omega$ and maps $\partial \Omega$ into
$\Bbb T$.

\item[\quad] (i)
Show that $0 \in f(\Omega)$.

\item[\quad] (ii)
Show that $f(\Omega) = \Bbb D$.

{\bf Hint:} To get ``$\supset$", apply (i) to $\phi \circ f$ for certian
holomorphic maps $\phi$ of $\Bbb D$ into $\Bbb D$.

\item[9.]
$f$ is continuous on $\overline{\Bbb D}$, holomorphic in $\Bbb D$ and
diam$f(\Bbb T) \leq 1$. Show that diam$f(r\Bbb T) \leq r$ for each
$0 \leq r \leq 1$.

{\bf Hint:}
diam$f(r \Bbb T) := \max \{|f(ru_1) - f(ru_2)| : u_1, u_2 \in \Bbb T\}$.
If this is achieved at $u_1, u_2$ consider the holomorphic function
$F(z) := f(zu_1) - f(zu_2)$.

\item[10.]
$h :\Bbb C \to \Bbb R$ is harmonic and non-constant.

\item[\quad] (i)
Prove that $h$ is not bounded above.

\item[\quad] (ii)
Prove that $h$ is not bounded below.

\item[\quad] (iii)
Prove that $h(\Bbb C) = \Bbb R$.





\end{description}    
%\end{Large}
\end{document}














