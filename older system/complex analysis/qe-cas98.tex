%S98 Complex Var QE-Bennett/Burckel

\def\hcorrection#1{\advance\hoffset by #1 }
\def\vcorrection#1{\advance\voffset by #1 }

\input amssym.def   %defines Bbb and gothic (frak)
\input amssym       %defines Bbb and gothic (frak)

\input prepictex.tex
\input pictex.tex
\input postpictex.tex

%\documentstyle[bbb]{report}
\documentclass[bbb]{report}

\vcorrection{-.7in}
\hcorrection{-.9in}

\topmargin  0in
\textwidth 6.4in
\textheight 10.0in

\mathsurround=2pt

\def\ds{\displaystyle}

\begin{document}

\pagestyle{empty}

\begin{large}

\begin{center}
COMPLEX VARIABLES QUALIFYING EXAMINATION - Spring 1998

(Bennett and Burckel)
\end{center}

\vspace{.3in}

\begin{description}

\item[] Let ${\Bbb R}$ denote the real line, ${\Bbb C}$ the complex
plane, ${\Bbb D}:=\{z\in{\Bbb C}:|z|<1\}$, $\Omega$ a non-void,
open, connected subset of ${\Bbb C}$, $C(\Omega)$ the continuous
${\Bbb C}$-valued functions on $\Omega$, $H(\Omega)$ the
complex-differentiable function on $\Omega$.

\vskip.2in

\item[1.] Let $S$ be the open square 
$]0,1[\times]0,1[$ and identify $(x,y)\in{\Bbb R}^2$
with $x+iy\in{\Bbb C}$.

\vspace{.3in}

\item[\qquad (i)] What does it mean for a function $f:S\to{\Bbb R}^2$
to be ${\Bbb R}$-differentiable \newline at $(x_0,y_0)\in S$?

\vspace{.3in}

\item[\qquad (ii)] If $f$ is ${\Bbb R}$-differentiable at
$(x_0,y_0)$, what property of its ${\Bbb R}$-derivative will make
$f$ also ${\Bbb C}$-differentiable at $x_0+iy_0$?

\vspace{.3in}

\item[2.] Suppose $\Omega$ is starlike with respect to its point $a$.
Show that for every $f\in H(\Omega)$ the companion function $F$ defined
by
$$ F(z):=\int_{[a,z]}f\qquad\forall z\in \Omega$$
is also holomorphic in $\Omega$ and satisfies $F'=f$.

\vspace{.3in}

\item[3.] What is the {\it topology of local uniform convergence}
in $C(\Omega)$?
Is this a metric topology? Show that:

\vspace{.3in}
\item[\qquad (i)] $C(\Omega)$ is complete in this topology.

\vspace{.3in}
\item[\qquad (ii)] $H(\Omega)$ is a closed subset of $C(\Omega)$.

\vspace{.3in}
\item[{\it Hint}:] For (ii) Morera's theorem is useful.

\vspace{.3in}

\item[4.] Prove that $f\mapsto f'$ is a continuous mapping of $H(\Omega)$
into itself (in the topology of Problem 3). Give an example of an
$\Omega$ for which this map is not surjective.

\item[{\it Hint}:] For the continuity, exploit Cauchy's integral
formula.

\vfill
\clearpage

\hfill page 2\par

\item[5.] Show that if $f\in H(\Omega)$ is one-to-one, then $f'$ is
zero-free in $\Omega$. Is the converse true?

\vspace{.3in}


\item[6.] $h:{\Bbb C}\to{\Bbb R}$ is harmonic and not constant. Prove
that $h$ has a zero.

\item[{\it Hint}:] If $h>0$ throughout ${\Bbb C}$, employ
Harnack's inequalities.

\vspace{.75in}
\item[] \begin{tabular}{lc}
{\bf 7.} Compute $\ds\int_{\Gamma}\ds\frac{z^2+1}{z^2-1}dz$, \\
where $\Gamma$ is the indicated path.
&
\begin{small}
\setlength{\unitlength}{0.015in}
\begin{picture}(100,50)(-150,0)    %(-200,35)
\put(0,-75){\line(0,1){150}} %y-axis
\put(-2.5,-50){\line(1,0){5}}
\put(7.5,-55){$-2i$}
\put(-2.5,-25){\line(1,0){5}}
\put(7.5,-30){$-i$}
\put(-2.5,25){\line(1,0){5}}
\put(7.5,20){$i$}
\put(-2.5,50){\line(1,0){5}}
\put(7.5,45){$2i$}

\put(-100,0){\line(1,0){200}} %x-axis
\put(-75,2.5){\line(0,-1){5}}
\put(-85,-15){$-3$}
\put(-50,2.5){\line(0,-1){5}}
\put(-60,-15){$-2$}
\put(-25,2.5){\line(0,-1){5}}
\put(-35,-15){$-1$}
\put(25,2.5){\line(0,-1){5}}
\put(22,-15){$1$}
\put(50,2.5){\line(0,-1){5}}
\put(46,-15){$2$}
\put(75,2.5){\line(0,-1){5}}
\put(71,-15){$3$}
\end{picture}
\end{small}
\end{tabular}

\vspace{2in}

\item[8.] The {\it cross-ratio} of an ordered quadruple of
distinct complex numbers is $[z_1,z_2,z_3,z_4]:
=\ds\frac{(z_1-z_2)(z_3-z_4)}{z_1-z_4)(z_3-z_2)}$.
Show that
$[z_1,z_2,z_3,z_4]=[w_1,w_2,w_3,w_4]$ if and only if there is a
M\"obius transformation (i.e., a linear fractional transformation)
that maps each $z_j$ to $w_j$.

\vspace{.3in}
\item[9.] Suppose $\ds\sum^\infty_{n=0}c_nz^n$ has radius of convergence
1. Show that the function $f(z):=\ds\sum^\infty_{n=0}c_nz^n$ which it
defines in ${\Bbb D}$ is holomorphic. Can you find such an
$f$ which can be continuously extended to $\overline{{\Bbb D}}$?
Disprove or give an example.

\vspace{.3in}
\item[10.] Prove that the zeros of a non-constant polynomial
depend continuously on its coefficients in the following sense: Given
$P(z)=c_0+c_1z+\dots +c_nz^n$ $(n>0,c_n\not=0)$ whose (distinct)
zeros are $z_1,\dots,z_r$ and given $\varepsilon>0$, there exists
$\delta>0$ such that whenever complex numbers satisfy
$|b_j-c_j|<\delta$ for all $j$, the polynomial
$Q(z):=b_0+b_1z+\dots+b_nz^n$ will have at least one zero in each of the
disks
\newline $D(z_j,\varepsilon):=\{z\in{\Bbb C}:|z-z_j|<\varepsilon\}$ and
all its zeros in the union
$\ds{r\atop{\bigcup\atop{j=1}}}D(z_j,\varepsilon)$ of these disks.



\end{description}

\end{large}

\end{document}
