% latex file
%Complex Analysis - Nagy & Burckel

\def\hcorrection#1{\advance\hoffset by #1 }
\def\vcorrection#1{\advance\voffset by #1 }

\input amssym.def   %defines Bbb and gothic (frak)
\input amssym       %defines Bbb and gothic (frak)

\input prepictex.tex
\input pictex.tex
\input postpictex.tex

%\documentstyle[bbb]{report}
\documentclass[bbb]{report}

\vcorrection{-.5in}
\hcorrection{-1in}

\topmargin  0in
\textwidth 6.4in
\textheight 9truein

\mathsurround=2pt

\def\ds{\displaystyle}

\def\R{{\Bbb R}}
\def\Z{{\Bbb Z}}
\def\C{{\Bbb C}}
\def\D{{\Bbb D}}


\begin{document}

\begin{Large}

\hfill Name \rule{2.5in}{.01in}
\par
\vspace{.25in}

\begin{center}
   COMPLEX VARIABLES QUALIFYING EXAM \\
   Fall 1998 \\
   (Burckel \& Nagy) \\
\end{center}


\vspace{.1in}

\begin{large}
{\bf
\noindent Do any 6 of the 8 problems. Standard notation throughout is:
$\Z$ is the set of integers; $\C$ is the set of complex numbers;
$\D=\{z\in\C:|z|<1\}$; $\Omega$ is a non-void open, connected
subset of $\C$.
$H(\Omega)$ is the set of all holomorphic functions in $\Omega$.
}
\end{large}

\vspace{.2in}


\begin{description}


\item[1.]
$f\in H(\Omega)$ is zero-free.

\vspace{.1in}

\item[\quad] (a)
Show that if $\Omega$ is convex, then $\exists g\in H(\Omega)$
such that $f=e^g$.

\vspace{.25in}

\item[\quad] (b)
Give an example of an $\Omega$ and an $f$ for which no such $g$ exists.

\vspace{.5in}

\item[2.]
Compute $\ds\int^\infty_0\ \ds\frac{dx}{1+x^n}$ for each integer $n\geq 2$.

\vspace{.5in}

\setlength{\unitlength}{.05in}
\begin{picture}(-20,0)(-80,0)
\put(-10,0){\line(1,0){30}}
\put(0,-10){\line(0,1){20}}
\put(0,0){\line(2,1){15}}

\begin{small}
\put(10,2){$\frac{2\pi}{n}$}
\end{small}

\put(-60,0){\underbar{Hint}: Use the contour}
\end{picture}

\vspace{.5in}

\item[3.]
$f:\D\times\D\to\C$ is bounded and has the property that
for each fixed $w\in\D$, $f(w,z)$ is a holomorphic function of
$z$, and for each $z\in\D$, $f(w,z)$ a holomorphic function of $w$.
Show that $f$ is (jointly) continuous on $\D\times\D$.

\item[\quad] \underbar{Hint}: Use Cauchy's integral representation in
both variables simultaneously.

\vspace{.5in}

\item[4.]
$f\in H(\C)$ and  $f(\C)\subset\{z\in\C:-1<Rez<1\}$.
Show that $f$ is constant.

\vfill
\pagebreak

\item[5.]
The usual topology on $H(\Omega)$ is the c.c. topology:
the topology of uniform convergence on compact subsets of $\Omega$.

\vspace{.1in}
\item[\quad] (a)
Show that $f\mapsto f'$ is a continuous mapping of $H(\Omega)$
into itself.

\vspace{.25in}
\item[\quad] (b)
Give an example of an $\Omega$ for which the mapping
$f\mapsto f'$ is not surjective.

\vspace{.5in}
\item[6.]
$f\in H(\Omega)$, $A$, $B$, $k$ are positive real constants and
$|f(z)|\leq A+B|z|^k$ for all $z$. Show that $f$ is a polynomial.

\vspace{.25in}

\item[7.] (a)
$P$, $Q$ are polynomials, $\deg(Q\geq 2+\deg(P)$,
$f=\frac{P}{Q}$, $S=\{p_1,\dots,p_k\}$ are the poles of $f$ and
they satisfy $S\cap \Z=\omega$. Show that
$$ \sum_{n\in\Z} f(n)=-\sum^k_{j=1} Res(g;p_j) $$
where $g(z)=\pi f(z)\cot(\pi z)$ for all $z\in\C\backslash\Z$.

\item{\quad} \underbar{Hint}: Consider integrating over the
contour which is a square having vertices
$\pm(n+\frac12)\pm(n+\frac12)i$, $n\in\Z$.

\vspace{.25in}
\item[\quad] (b)
Use a variation on the idea in (a) to compute $\ds\sum^\infty_{n=1}
\ \ds\frac{1}{n^2}$.

\vspace{.25in}
\item[8.]
For $f:\Omega\to\C$ three fundamental properties are equivalent:

\vspace{.1in}
\item[\quad] (a)
$f$ is locally of power-series type.

\vspace{.1in}
\item[\quad] (b)
$f$ satisfies Cauchy's integral formula for circles.

\vspace{.1in}
\item[\quad] (c)
$f$ satisfies the Cauchy-Riemann equation(s).

\item{\quad} State these properties precisely and prove one of the
six implications.

\vfill



\end{description}

\end{Large}

\end{document}
