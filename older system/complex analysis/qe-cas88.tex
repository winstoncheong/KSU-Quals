% latex file
\def\hcorrection#1{\advance\hoffset by #1 }
\def\vcorrection#1{\advance\voffset by #1 }

\documentclass{article}
\usepackage{my,amsxtra,amssymb,amsthm}

\vcorrection{-1.0in}
\hcorrection{-0.8in}
\textwidth 6.0in
\textheight 9.0in

\def\R{\mathbb R}
\def\C{\mathbb C}
\def\N{\mathbb N}
\def\Z{\mathbb Z}
\def\Q{\mathbb Q}

\begin{document}
%\begin{Large}






\begin{center}\begin{LARGE}
{\bf Complex Analysis Qualifying Exam}\\ 
{\bf Spring 1988}\\ \end{LARGE}
\end{center}
\vspace{0.1in}
\noindent\hrulefill\\

Throughout ${\bf C}$ denotes the complex plane and ${\bf D} := \{z \in {\bf C}: |z| < 1\}$
the open unit disc therein. ${\bf N}$ is the positive integers, ${\bf Z}$ all the
integers and ${\bf R}$ the real numbers. $I$ denotes the identity function:
$I(z) = z$ for all $z$.

\begin{description}
\item[1.]
$f$ is holomorphic and bounded in ${\bf D}$ and $f(z) \to 0$ as
$z \to 1$ along the upper arc of the circle $|z - \frac{1}{2}| = \frac{1}{2}$.
Show that $f(x) \to 0$ as $x \to 1$ along $[0,1[$.
{\bf Hints:} For each $N \in {\bf N}$ let $f_N (z) := z^N f(z)$,
$F_N(z) := f_N (z) \overline{f_N (\overline{z})}$. Show that $F_N$
is holomorphic and for $\varepsilon >0$ there is an $N$ such that
$|F_N| \leq \varepsilon$ on the whole circle
$|z - \frac{1}{2}| = \frac{1}{2}$, and infer that
$|f(x)|^2 \leq \varepsilon x^{-2 N}$ for $x \in ]0,1[$ by the Maximum
Modulus Principle.

\item[2.]
$f$ is holomorphic and bounded by $M$ in ${\bf D}$ and has zeros at the distinct
points $a_1, \dots, a_N \in {\bf D}$. Prove that
$$|f(z)| \leq M \prod^N_{j=1} \left|\frac{z-a_j}{1- \overline a_j z} \right|
  \quad \forall z \in {\bf D}.$$
Is this an improvement over the hypothesized inequality $|f(z)| \leq M$?
{\bf Hint:} $f(z) \prod^N_{j=1} \frac{1-\overline a_j z}{z-a_j}$, appreance
to the contrary, is holomorphic in ${\bf D}$ and bounded by $M$. Why?

\item[3.]
If $f$ is {\it one-to-one} and holomorphic in the open set $U$ except for
isolated singularities, then $f$ has no essential singularities and at most
one pole.

\item[4.]
$f$ is continuous in ${\bf D}$ and holomorphic in ${\bf D} \backslash [-1,1]$. Show
that $f$ is actually holomorphic in ${\bf D}$.

\item[5.]
Show that if $c$ is a non-removable isolated singularity of the holomorphic
function $f$, then $c$ is an essential singularity of the function $e^f$.

\item[6.]
Let ${\bf A}$ denote the {\it disc algebra}: $C(\overline {\bf D}) \cap H ({\bf D})$.
Find all the homomorphisms $\phi$ of ${\bf A}$ into ${\bf C}$.

{\bf Hints:} The number $c := \phi (I)$ plays a special role. Show that for
each $f$ the number $\phi (f)$ lies in $f(\overline {\bf D})$ and determines
$\phi$ first on the polynomials.

\item[7.]
State necessary and sufficient conditions on a sequence
$(b_n)_{n \in N} \subset {\bf D}$ in order that

(i) there exists a non-zero holomorphic function on ${\bf D}$ with zeros at each
$b_n$;

(ii) there exists a bounded, non-zero holomorphic function on ${\bf D}$ with
zeros at each $b_n$.

In case (ii) how would you construct such a function (outline only, no proof).

\item[8.]
Let $\Omega$ be an open connected subset of ${\bf C}$, $f_n$ holomorphic in
$\Omega$, and suppose that $\{f_n\}$ converges to a non-constant function
$f$ uniformly on each compact subset of $\Omega$. Show that

\item[\quad] (i)
$\overline \lim_{n \to \infty} f_n (K) \subset f(K)$ for every compact
$K \subset \Omega$ and

\item[\quad] (ii)
$\underline {\lim}_{n \to \infty} f_n (G) \supset f(G)$ for every open
$G \subset \Omega$.

Suppose in addition that there exists an $M < \infty$ such that every
$w \in {\bf C}$ and $n \in {\bf N}$
$$\hbox{card\ } [f^{-1}_n (w) ] \leq M.$$

Show that then

\item[\quad] (iii)
$\hbox{card\ } [f^{-1} (w)] \leq M$ for all $w  \in {\bf C}$.

\item[9.]
Let $f$ be holomorphic in a neighborhood of $\overline D$. Suppose
$z_0 \in \partial {\bf D}$ satisfies $|f(z_0)| = \max |f(\overline D)|$. Show that
$f^\prime (z_0) \neq 0$, unless $f$ is constant in ${\bf D}$.
{\bf Hints:} WLOG, $1 = z_0 = |f(z_0)|$. Then $f$ non-constant plus the
Maximum Modulus Principle gives $f({\bf D}) \subset {\bf D}$. If $f(0) = 0$, then
use Schwarz to argue
$\left|\frac{f(1)-f(x)}{1-x} \right| \geq \frac{1-x}{1-x} = 1$
for $x \in [0,1[$, so $|f^\prime (1)| \geq 1$. In general, let
$a = f(0) \in {\bf D}$, form $T(w) := \frac{w-a}{1-\overline a w}$,
$F := T \circ f$ and apply the preceding. Compute $F^\prime (1)$ by the
chain rule and express $f^\prime (1)$ in terms of it.

\item[10.] (i)
State the Harnack inequalities for harmonic functions.

\item[\quad] (ii)
Let $h :{\bf C} \to {\bf R}$ be harmonic and non-constant. Show that $h$ has at least
one zero. {\bf Hint:} One method is to use (i).

\item[\quad] (iii)
Let $h : {\bf C} \to {\bf R}$ be harmonic and non-constant. Show that
$h({\bf C}) = {\bf R}$. {\bf Hint:} Use (ii). Alternatively, since $h({\bf C})$ is
necessarily an interval in ${\bf R}$, if not all of ${\bf R}$, it is bounded above
or below and with harmonic conjugate of $h$ we should be
able to use Liouville.





\end{description}    
%\end{Large}
\end{document}














